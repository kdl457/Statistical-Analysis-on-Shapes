\documentclass[a4,danish]{article}

\usepackage{amssymb}
\usepackage{amsmath}
\usepackage{xcolor}
\usepackage{soul}
\usepackage{enumerate}

\newcommand{\Z}{\mathbb{Z}}
\newcommand{\Q}{\mathbb{Q}}
\newcommand{\R}{\mathbb{R}}
\newcommand{\N}{\mathbb{N}}
\newcommand{\C}{\mathbb{C}}
\renewcommand{\S}{\mathbb{S}}
\renewcommand{\P}{\text{P}}

\renewcommand{\phi}{\varphi}
\renewcommand{\epsilon}{\varepsilon}

\newcommand*\diff{\mathop{}\!\mathrm{d}}

\setlength{\parskip}{1em}
\setlength{\parindent}{0em}

\title{The tangent space of closed curves in $\R^2$}
\author{Mads and Anders}
\date{\today}

\begin{document}
% \maketitle

\section*{The tangent space of closed curves in $\R^2$}
\label{sec:tangent-space-closed}


\paragraph{Constructing the space of closed curves.}
Intuitively, we want to consider the space of all (smooth) closed
curves in $\R^2$. This can be seen as the space of all submanifolds in
$\R^2$ which are diffeomorphic to the unit circle $\S^1$. If we let
$\text{Imm}(\S^1,\R^2)$ denote the space of all \textit{immersion}
from the inut circle into the plane, we can define the space we
want to consider as
\begin{equation*}
  B :=
  \left\{
    q(\S^1) \mid q \in \text{Imm}(\S^1,\R^2)
  \right\}.
\end{equation*}
Here we simply think of $q(\S^1) \subset \R^2$ as a subspace and forget
about the actual map $q$. (Keeping this mapping in mind, we could
define the space in another way; but this is not so important right
now.)

\paragraph{The tangent space of B.}
For ordinary finite dimensional manifolds $B$, Lee defines the tangent
space at a point $p \in B$ through the notation of
\textit{derivations}; this is a rather abstract construction, but is
nice to work with. Using this, one can define the notion of a
tanget vector to a curve in $B$ passing through the point $p$. Then,
one can define an equivalence relation on the space of such curves and
obtain an equivalent definition of the tanget space, which is more
intuitive. One can also work the other way around

...

Imagine for a moment that
one way to define the tangent space $T_p$ at a point $p \in B$, is to
consider tangent vectors to curves/paths passing through the point
$p$. If $\gamma \colon [0,1] \rightarrow B$ is some path in $B$ we
consider
\begin{equation*}
  \gamma'(x)
  % := \frac{\partial }{\partial t} \gamma(t) \bigg\rvert_{t=x}
  := \frac{\partial }{\partial t} \bigg\rvert_{t=x} \gamma(t),
\end{equation*}
and define and equivalence relation as $\gamma_1 \sim \gamma_2$
whenever $\gamma'_1(0)=\gamma'_2(0)$

...

Taking for granted that $B$ actually is a (Fr\'echet) manifold, we now
want to determine what the tangent spaces look like.

\paragraph{Question / considerations / imprecisions.}






\end{document}