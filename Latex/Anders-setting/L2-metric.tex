\subsection*{The L2 metric vanishes}
\label{sec:l2-metric-vanishes}

\hl{Some introductory text...}

Definition of the length of the path $\phi$ directly from the path alone:
\begin{equation*}
  L(\phi) = \int_{0}^{1} \int_{\S^{1}}
  \frac{\langle\phi_{t},i \phi_{\theta}\rangle^2}{|\phi_{\theta}|}  \diff \theta \diff t
\end{equation*}
Rewrite this as
\begin{equation*}
  \int_{0}^{1} \int_{\S^{1}}
  \left(
    \frac{\langle\phi_{t},i
      \phi_{\theta}\rangle}{|\phi_{t}||\phi_{\theta}|}
  \right)^2
  |\phi_t|^2   |\phi_{\theta}|
  \diff \theta \diff t
  =
  \int_{0}^{1} \int_{\S^{1}}
  \cos(\alpha(\phi_t, i\phi_{\theta}))^2
  |\phi_t|^2   |\phi_{\theta}|
  \diff \theta \diff t,
\end{equation*}
with $\alpha(x,y)$ denoting the angle between $x$ and $y$. When
constructing a zigzag-path the angle will be \hl{(approximately?)}
constant in $\theta$ and $t$,
and it is given by
\begin{equation*}
  n = \tan(\alpha).
\end{equation*}
We have that
\begin{equation*}
  \cos(\arctan(n)) = (1+n^2)^{-1/2}
  = O(n^{-1}),
\end{equation*}
so we can write
\begin{equation*}
  L(\phi) = O(n^{-2})
  \int_{0}^{1} \int_{\S^{1}}
  |\phi_t|^2   |\phi_{\theta}|
  \diff \theta \diff t,
\end{equation*}
in this case. To show that such a zigzag path has arbitrary
small length we just need to show that the integral does not grow
faster than $n^2$.

As an example, take the simply case where we expand the circle $e^{i\pi\theta}$ to
$2e^{i\pi\theta}$. The zigzag path in then concretely given as
\begin{equation*}
  \phi(t,\theta) = e^{i\pi\theta}
    \sum_{k=0}^{n-1}
    h^{n,k}(t,\theta) + g^{n,k}(t,\theta)
\end{equation*}
where
\begin{equation*}
  \begin{aligned}
    h^{n,k}(t,\theta) & := 1_{[\frac{k}{n},\frac{k}{n} +
      \frac{1}{2n})}(\theta) \left( 1+2t(n\theta-k) \right), \\
    g^{n,k}(t,\theta) & := 1_{[\frac{k}{n} + \frac{1}{2n},\frac{k+1}{n})}(\theta)
    \left( 1+2t(1-n\theta-k) \right).
  \end{aligned}
\end{equation*}

We have that
\begin{equation*}
  \begin{aligned}
    |\phi_t| & = \sum_{k=0}^{n-1} |h^{n,k}_t| + \sum_{k=0}^{n-1}
    |g^{n,k}_t|, \\
    |\phi_{\theta}| & = \sum_{k=0}^{n-1} |h^{n,k}_{\theta} + h^{n,k} | +
    \sum_{k=0}^{n-1}
    |g^{n,k}_{\theta} + g^{n,k} |,
  \end{aligned}
\end{equation*}
so by symmetry
\begin{equation*}
  \begin{aligned}
    \int_{0}^{1}
    |\phi_t|^2   |\phi_{\theta}|
    \diff \theta
    & =
    2n \int_{0}^{\frac{1}{2n}} |h^{n,0}_t|^2 |h^{n,0}_{\theta} + h^{n,0} |
    \diff \theta \\
    & = 2n \int_{0}^{\frac{1}{2n}}
    (2n\theta)^2(2tn+1+t2n\theta) \diff \theta \\
    & = \int_{0}^1
    u^2(2tn+1+t u) \diff \theta \\
    & = O(n),
  \end{aligned}
\end{equation*}
for $t\in[0,1]$ which gives the result.

%%% Local Variables:
%%% mode: latex
%%% TeX-master: "local-masters/local-main-L2-metric"
%%% End:
