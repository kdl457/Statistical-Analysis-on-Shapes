\section{The manifold of curves}
\label{sec:manifold-curves}

A 2-dimensional shape can be thought of as a closed (smooth) curve in $\R^2$. Thus, we want to define a manifold structure on the space of these curves. Doing this mathematically correct is rather technical because we need to consider quotient spaces of infinite dimensional manifolds. In our exposition we shall to a large extend ``define our way out of this'' by, for example, defining tangent vectors of curves instead of deducing how these look like from the formal definition of the underlying manifold. We shall motivate our definitons geometrically and then deduce some properties from these definitons -- properties, which can also be deduced from the formal definitions. At the end of this section we briefly address what we miss with our more informal treatment.

A shape in $\R^2$ can either be thought of as a parametrized object, e.g., as a function $\S^1 \ni \theta \mapsto c(\theta) \in \R^2$; but it can also be thought of as an unparametrized object, e.g., the \textit{image} of such a function $\text{Im}(c) \subset \R^2$ (as illustrated in Figure~\ref{fig:circle-mapping}). We impose some smoothness structure on the curves and define the spaces we want to consider formally as
\begin{align*}
  \I & := \I(\S^1, \R^2)
       :=  \left\{ c\colon \S^1 \rightarrow \R^2 \mid c \text{ is an immersion} \right\} \\
     & \quad \quad = \left\{c\colon \S^1 \rightarrow \R^2 \mid \forall \theta \in S^1: \, c_{\theta} \not = 0 \right\}, \\
  \mathcal{I}
     & := \mathcal{I}(\S^1, \R^2)  := \I(\S^1, \R^2)/\mathrm{Diff}(\S^1).
\end{align*}
$\I$ is the space of \textit{immersion} of the unit circle into $\R^2$, and $\mathcal{I}$ is then this space modulo reparametrization, i.e., we identify two objects $a,b \in \I$ in $\mathcal{I}$ if $a=b \circ \phi$, where $\phi \in \mathrm{Diff}(\S^1)$ is a diffeomorphism on the unit circle.
\footnote{Actually, to make the construction work, we need to restrict the space of immersions to the set of all \textit{free} immersions. This is a minor technical detail so we skip these concerns for now. We mention them briefly again in Section~\ref{sec:hlcomp-this-form}.}

When our interest is upon the \textit{shape}, we are not really interested in the underlying parametrization of this shape, and so we are mostly interested in the space $\mathcal{I}$. However, it is easiest to construct a manifold structure on $\I$ and then deduce one on $\mathcal{I}$, so we start by considering the parametrized space of immersions.

\begin{figure}
  \centerline{\includegraphics[width=0.9\linewidth]{circle_mapping.pdf}}
  \caption{A simple shape. We can think of a shape as the whole
    mapping, or simply as the subset to the right.}
  \label{fig:circle-mapping}
\end{figure}

\subsection{The manifold of parametrized curves -- $\I(\S^1, \R^2)$}
\label{sec:parametrized-curves}

The first structure we want to impose on our manifold of curves is tangent spaces and tangent vectors to our elements in the space.

For ordinary finite dimensional manifolds $M$ we have worked with so far, tangent vectors are defined as \textit{derivatives}. This is a rather abstract construction, which, however, turns out to be nice to work with. Fortunalety, we know that this definition corresponds to the more geometrically intuitive definition of tangent vectors at a point $m \in M$ as derivatives of paths going through $m$. We shall use this as motivation for our definition of tangent vectors to points in $\I$.

Consider a point $c\in \I$ and a path in $\I$ defined around $0$ that goes through the point $c$. This path is a map
\begin{equation}
  \label{eq:path-in-imm}
  [-\epsilon, \epsilon] \ni t \mapsto  q(t, \blank) :=  (\theta\mapsto q(t, \theta)) \in \I,
  \quad q(0,\theta) = c(\theta),
\end{equation}
i.e., for each $t$ we get a parametrized curve, and at 0 we get the curve $c$. We can also think of this as a (smooth) map
\begin{equation*}
  [-\epsilon, \epsilon] \times \S^1  \ni (t, \theta) \mapsto q(t, \theta) \in \R^2.
\end{equation*}
As in the finite dimensional case we can now take the \textit{time derivate} of our path and evaluate this at 0; of course, this is technically not obvious, as our constructed path maps into a function space, but we shall here simply take this derivative to be understood pointwise:

\begin{definition}
  \label{def:tangt-space-of-curves}
  The tangent space $T_{c}\I$ to an element $c \in \I$ consists of all functions
  $h\colon \S^1 \rightarrow \R^2$ such that
  \begin{equation*}
    h(\theta) = q_t(0, \theta) := \frac{\partial }{\partial t} \bigg\vert_{t=0} q(t,\theta),
  \end{equation*}
  where $q$ is some path passing through $c$ at 0 (as defined in \eqref{eq:path-in-imm}).
\end{definition}

By this definition, we can essentially think of tangent vectors in $\I$ as vector fields on the circle. Figure~\ref{fig:def-tang-imm} illustrates the idea behind the definition.

\begin{figure}
  \centering
  \begin{subfigure}{.49\textwidth}
    \centering
    \includegraphics[width=1\linewidth]{path.pdf}
  \end{subfigure}
  \begin{subfigure}{.49\textwidth}
    \centering
    \includegraphics[width=0.8\linewidth]{circle_vectorfield.pdf}
  \end{subfigure}
  \caption{Illustration of tangent vectors in $\I$. The left illustration shows a path in the space of curves and how to obtain a tangent vector from this. The right illustration shows how to think of this tangent vector as a vector field on the circle.}
  \label{fig:def-tang-imm}
\end{figure}

We want to make our space of curves into a Riemmanian manifold so the next step is to impose a metric on the manifold via an inner product on the tangent spaces. Because the tangent spaces are function spaces, the most obvious metric to use is some version of the $\mathrm{L}^2$ metric.

\begin{definition}
  The \textit{$\mathrm{G}$ metric} or \textit{$\mathrm{G}_c$ metric} at the point $c \in \I$ is defined as
  \begin{equation*}
    \mathrm{G}_c(h,k) :=
    \left(
      \int_{\S^{1}} \left\langle{h(\theta)
          , k(\theta)}\right\rangle |c_{\theta}| \diff \theta
    \right)^{\frac{1}{2}},
  \end{equation*}
  $h,k \in T_cB = C^{\infty}(\S^1,\R^2)$.
\end{definition}

Adding the parameter derivative $c_{\theta}$ makes the metric invariant to reparametrizations which is essential when we want to let this fall down as a metric on $\mathcal{I}$. From this definition it is straightforward to define a notion of \textit{length} of a path in $\I$ which again allows us to define a \textit{distance} between two point in our space of curves. By defintion, the pointwise time derivative at $t$ of a path $t \mapsto q(t, \blank)$ in $\I$ is a tangent vector to the curve $q(t,\blank)$, so the following definition makes sense.

\begin{definition}
  \label{def:length-in-imm}
  Let $t \mapsto q(t,\blank)$ be a path in $\I$ with $t \in [0,1]$. The \textit{length} of the path $q$ with respect to the $\mathrm{G}$ metric  is
  \begin{equation*}
    L(q) := \int_{0}^{1} \mathrm{G}_{q(t,\blank)}(q_t,q_t) \diff t =
    \int_{0}^{1}
    \left(
      \int_{\S^{1}} \|q_t\|^2 |q_{\theta}| \diff \theta
    \right)^{\frac{1}{2}}
    \diff t.
  \end{equation*}
  The \textit{geodesic} (or shortest) distance between two curves $b,c \in \I$ (with respect to the $\mathrm{G}$ metric) is
  \begin{equation*}
    D(b,c) := \inf_{q \in \mathcal{Q}} L(q),
  \end{equation*}
  where $\mathcal{Q}$ denotes all paths $q$, such that $q(0,\blank)=b$ and $q(1,\blank)=c$.
\end{definition}


\subsection{The manifold of unparametrized curves -- $\mathcal{I}(\S^1, \R^2)$}
\label{sec:manif-unpar-curv}

Using the setup from the previous section, we now define a manifold structure on $\mathcal{I}$. As it is easiest to represent elements of this space with elements from $\I$, we are particularly interested in how to calculate the length of a path in $\mathcal{I}$ directly from a parametrized representative of this path in $\I$. For a parametrized curve $c \in \I$ we write $\pi(c) := \mathrm{Im}(c)\in \mathcal{I}$ for the projection onto the quotient space, and we refer to $c\in\I$ as a \textit{(parametrized) representative} for $\pi(c) \in \mathcal{I}$. Similarly for a path $q = q(t, \blank) \in \I$ we construct the projection $\pi(q) = \pi(q(t,\blank)) \in \mathcal{I}$ and refer to $q$ as a representative for $\pi(q)$.

First of all we need to know how the tangent vectors of the quotient space $\mathcal{I}$ look like. We cannot directly use the same approach as in the previous subsection, because our definition of tangent vectors would then be
sensitive to reparametrizations: Consider a time-dependent
reparametrization $\phi(t,\theta)$, or, equivalently, a path $t \mapsto \phi(t,
\blank)$ in $\text{Diff}(\S^1)$; then the two paths $t \mapsto
\pi(q(t, \blank))$ and $t \mapsto \pi(q(t, \phi(t, \blank)))$ are
identical in $\mathcal{I}$ but give rise to two different vector fields.

To make better sense of the tangents vectors of the quotient space, we
use the following result, with $q \circ \phi$ meaning $q(t,\phi(t,\theta))$ to ease notation.

\begin{proposition}
  \label{prop:horizontal-path}
  For every path $t \mapsto q(t, \blank)$ in $\mathrm{Imm}$ there exists a
  time-dependent reparametrization $t \mapsto \phi(t, \blank) \in
  \mathrm{Diff}(\S^{1})$ such that the path
  \begin{equation*}
    t \mapsto \tilde{q}(t, \theta):=q(t, \phi(t,\theta))
  \end{equation*}
  fulfills
  $\langle \tilde{q}_t, \tilde{q}_{\theta}\rangle=0$ for all $(t,\theta) \in [0,1]\times \S^1$, and such that $\phi(0, \theta)=\theta$. Furthermore, it holds that
  \begin{equation}
    \label{eq:canon-repar}
    \phi_t = a \circ \phi
    = -\frac{\langle q_t \circ \phi, q_{\theta} \circ \phi\rangle}{|q_{\theta}\circ \phi|^2},
    \quad a := -\frac{\langle q_t,
      q_{\theta}\rangle}{|q_{\theta}|^2}.
  \end{equation}
\end{proposition}

\begin{proof}
  See section $2.5$ of \cite{michor2003riemannian} 
\end{proof}

\begin{remark}
  \label{remark:ortho-decom}
  \begin{enumerate}
  \item  When we write $\tilde{q}$ in the following, we shall we refer to a path obtained from another path $q$ by reparametrizing with $\phi$ above.
  \item Note that for every vector field $h \in T_c(\I)$, determined from the path $q$, we can make a pointwise decomposition of $h$ onto $q_{\theta}(0, \blank)$ and $iq_{\theta}(0, \blank)$ by using the pointwise orthogonal projection. Explicitly we have that
  \begin{equation*}
    h = q_t = p_{q_{\theta}}(q_{t}) + p_{iq_{\theta}}(q_{t}),
  \end{equation*}
  where $p$ is taken to be the standard \textit{pointwise} $\R^2$ orthogonal projection, which is given as
  \begin{equation*}
    p_v(u) = \frac{\langle v, u \rangle}{|v|^2} v, \quad u, v \in \R^2.
  \end{equation*}
  More correctly we should thus write
  \begin{equation*}
    h(\theta) = q_t(0,\theta) = p_{q_{\theta}(0,\theta)}(q_{t}(0,\theta)) +
    p_{iq_{\theta}(0,\theta)}(q_{t}(0,\theta)).
  \end{equation*}
  From this we see that the time derivative of the reparametrization in the previous Proposition is the coefficient function for the projection onto the parameter derivative of the original path $q$; this becomes relevant in a moment.
\end{enumerate}
\end{remark}

We can use this result to define tangent vectors to elements of $\mathcal{I}$ in
a consistent way:

\begin{definition}
  \label{def:tang-quotient}
  A \textit{tangent vector} $h$ to an element $\pi(c) \in \mathcal{I}$ is defined as a vector field obtained from some path $(-\epsilon, \epsilon) \ni t \mapsto q(t, \blank) \in \I$, with $q(0,\blank) = c$, by
  \begin{equation}
    \label{eq:tang-quotient}
    h(\theta) = \frac{\partial }{\partial t} \bigg\vert_{t=0} \tilde{q}(t,\theta)
    = \frac{\partial }{\partial t} \bigg\vert_{t=0} q(t, \phi(t,\theta)),
  \end{equation}
  with $\tilde{q}$ and $\phi$ given in accordance with remark~\ref{remark:ortho-decom}.
\end{definition}

First we note that this gives us the following visualization of the tangents spaces of $B$.

\begin{proposition}
  \label{prop:tangent-space-orthogonal}
The tangent space to an element $\pi(c) \in \mathcal{I}$ consists of orthonormal vector fields on the circle, i.e.,
  \begin{equation*}
    T_{\pi(c)}(\mathcal{I}) =
    \left\{
      b i c_{\theta} \mid b \in C^{\infty}(\S^1,\R)
    \right\}.
  \end{equation*}
\end{proposition}

\begin{proof}
  This follows from Definition~\ref{def:tang-quotient} and the property of the reparametrization $\phi$.
\end{proof}

As the length of a path in $\mathcal{I}$ is our primary concern, we skip straight to this without actually defining the inner product on the tangent spaces. The central idea is to mimic Definition~\ref{def:length-in-imm} on the reparametrized path $\tilde{q}$; and though we don't bother to go trough a inner product on the tangent spaces, we note that by this construction the time derivative of the path $\tilde{q}$ is a valid tangent vector in $\mathcal{I}$ at every point $\pi(q(t,\blank))$

\begin{definition}
  The \textit{length} in $\mathcal{I}$ of a path $\mathrm{q}=\pi(q)$ is
  \begin{equation*}
    \mathcal{L}(\mathrm{q})= \mathcal{L}(\pi(q)) := L(\tilde{q}) =
    \int_{0}^{1}
    \left(
      \int_{\S^{1}} \|\tilde{q}_t\|^2 |\tilde{q}_{\theta}| \diff \theta
    \right)^{\frac{1}{2}}
    \diff t,
  \end{equation*}
  with $\tilde{q}(t,\theta)=q(t,\phi(\theta))$ as in remark~\ref{remark:ortho-decom}.
  The \textit{geodesic distance} between to shapes $\mathrm{b}, \mathrm{c} \in \mathcal{I}$ represented by $c,b \in \I$, is
  \begin{equation*}
    \mathcal{D}(\mathrm{b},\mathrm{c}) = \mathcal{D}(\pi(b),\pi(c)) := \inf_{q \in \mathrm{Q}} \mathcal{L}(q),
  \end{equation*}
  where $\mathrm{Q}$ denotes all paths $\mathrm{q}$ in $\mathcal{I}$, such that $\mathrm{q}(0)=\pi(b)$ and $\mathrm{q}(1)=\pi(c)$.
\end{definition}

\begin{proposition}
  \label{prop:length-quotient}
    $\mathcal{L}$ is well-defined, and for any representative $t \mapsto q(t, \blank) \in \I $ of the path $t \mapsto \tilde{q}(t) \in \mathcal{I}$ the length can be calculated as
    \begin{equation}
      \label{eq:length-quotient}
    \mathcal{L}(\tilde{q}) = \int_{0}^{1}
    \left(
      \int_{\S^1}  \frac{\langle q_t, i q_{\theta}\rangle^2}{|q_{\theta}|} \diff \theta
    \right)^{\frac{1}{2}} \diff t.
  \end{equation}
\end{proposition}

\begin{proof}
  Again, to ease notation we write $q \circ \phi$ to mean $q(t, \phi(t,\theta))$ and so on during this proof.
  First, we shows that \eqref{eq:length-quotient} implies that $\mathcal{L}$ is well-defined; so assume \eqref{eq:length-quotient} holds and let $q(t,\blank)$ and $p(t, \blank)$ be two different representatives for $\tilde{q}(t)$. This means that we must have a reparametrization $\psi(t,\theta)$ such that
  \begin{equation*}
    p(t, \psi(t,\theta)) = q(t, \theta).
  \end{equation*}
  Then
  \begin{equation*}
    p_t = q_t\circ \psi + \psi_t (q_t \circ \psi), \quad
    p_{\theta} = \psi_{\theta}(q_{\theta} \circ \psi),
  \end{equation*}
  so
  \begin{equation*}
    \begin{aligned}
      \left\langle
        p_t, i p_{\theta}
      \right\rangle
     &  =     \left\langle
        q_t \circ \psi + \psi_t (q_{\theta} \circ \psi),
        \psi_{\theta} (iq_{\theta} \circ \psi)
      \right\rangle \\
     & =     \left\langle
        q_t \circ \psi,
        \psi_{\theta} (iq_{\theta} \circ \psi)
      \right\rangle \\
      & = (\langle q_t, i q_{\theta}\rangle \circ\psi) \psi_{\theta},
    \end{aligned}
  \end{equation*}
  and thus
  \begin{equation*}
    \int_{\S^1} \frac{\langle p_t, i p_{\theta}\rangle^2}{|p_{\theta}|} \diff \theta
    =
    \int_{\S^1}
    \left(
      \frac{\langle q_t, i q_{\theta}\rangle^2}{|q_{\theta}|}
    \right) \circ \psi |\psi_{\theta}| \diff \theta
    =    \int_{\S^1}
      \frac{\langle q_t, i q_{\theta}\rangle^2}{|q_{\theta}|}  \diff \theta,
    \end{equation*}
    which shows that the length does not depend on the parametrization of the path.

  Next, by construction, the tangent vectors along the path $\tilde{q}$ in $B$ is given as
  \begin{equation*}
    \frac{\partial }{\partial t}  (q \circ \phi)
    = q_t \circ \phi + \phi_t (q_{\theta} \circ \phi)
  \end{equation*}
  Now, as in remark~\ref{remark:ortho-decom}, decompose $q_t \circ \phi $ by projecting pointwise onto $q_t \circ \phi $ and $i q_t \circ \phi $. Then we get
  \begin{equation*}
    q_t \circ \phi =
    % \frac{\langle q_t \circ \phi, iq_{\theta} \circ \phi \rangle}
    % {|q_{\theta} \circ \phi|^2}(iq_{\theta} \circ \phi) +
    % \frac{\langle q_t \circ \phi, q_{\theta} \circ \phi \rangle}
    % {|q_{\theta} \circ \phi|^2}(q_{\theta} \circ \phi),
    % =
    \left(
      \frac{\langle q_t  , iq_{\theta}   \rangle}
    {|q_{\theta}  |^2}iq_{\theta}
  \right) \circ \phi +
  \left(
    \frac{\langle q_t  , q_{\theta}   \rangle}
    {|q_{\theta}  |^2}q_{\theta}
  \right) \circ \phi,
  \end{equation*}
  and by Proposition~\ref{prop:horizontal-path} we see that the last term cancels with
  $\phi_t(q_{\theta}\circ \phi)$, so
  \begin{equation*}
    \frac{\partial }{\partial t} ( q \circ \phi )=
    % \frac{\langle q_t \circ \phi, iq_{\theta} \circ \phi \rangle}
    % {|q_{\theta} \circ \phi|^2}(iq_{\theta} \circ \phi)
    % =
    \left(
      \frac{\langle q_t  , iq_{\theta}   \rangle}
      {|q_{\theta}  |^2}iq_{\theta}
    \right)
    \circ \phi.
  \end{equation*}
  For any fixed $t \in [0,1]$, the reparametrization $\phi$ is just an ordinary reparametrization of the curve $\theta \mapsto q(t,\theta)$, so by invariance of the metric we have that
  \begin{equation*}
    \begin{aligned}
    & G_{q(t,\phi(t,\blank))}^2
    \left( \frac{\partial }{\partial t} ( q \circ \phi ),
      \frac{\partial }{\partial t} ( q \circ \phi )
  \right) \\
  & \quad=  G_{q(t,\phi(t,\blank))}^2
    \left(
          \left(
      \frac{\langle q_t  , iq_{\theta}   \rangle}
      {|q_{\theta}  |^2}iq_{\theta}
    \right)
    \circ \phi,
        \left(
      \frac{\langle q_t  , iq_{\theta}   \rangle}
      {|q_{\theta}  |^2}iq_{\theta}
    \right)
    \circ \phi
    \right) \\
    & \quad =
    G_{q(t,\blank)}^2
    \left(
      \frac{\langle q_t  , iq_{\theta}   \rangle}
      {|q_{\theta}  |^2}iq_{\theta} ,
      \frac{\langle q_t  , iq_{\theta}   \rangle}
      {|q_{\theta}  |^2}iq_{\theta}
    \right) \\
    & \quad =
    \int_{\S^1}
    \left\|
      \frac{\langle q_t  , iq_{\theta}   \rangle}
      {|q_{\theta}  |^2}iq_{\theta}
    \right\|^2 |q_{\theta}  | \diff \theta \\
    & \quad =
    \int_{\S^1}
      \frac{\langle q_t  , iq_{\theta}   \rangle^2}
      {|q_{\theta}  |}  \diff \theta,
  \end{aligned}
\end{equation*}
from which \eqref{eq:length-quotient} follows immediately by definition.
\end{proof}

The above result shows that our definition of the length of a path in the shape space $\I$ is meaningful -- at least at first sight. However, it turns out that this metric vanishes on all of $\mathcal{I}$.
The result above is still very important, because the formula for directly calculating the length of a path from some parametrized representation of it is essential in showing how the $\mathcal{L}^2$-metric breaks down, as we will see in Section~\ref{sec:l2-metric-vanishes} below.

\subsection{Some notes about a more formal approach}
\label{sec:hlcomp-this-form}

Before moving on, we pause to make a few comparisons with our approach above in constructing the shape space $\mathcal{I}$ and a more formal one based on some general results of infinite-dimensional manifolds and the diffeomorphism group.

What characterises a finite-dimensional manifold is that it can locally be mapped homeomorphically to $\R^n$. As every finite-dimensional vector space is isomorphic to $\R^n$, this is the obvious space to use as a model space. For infinite-dimensional vector spaces there is no longer any ``obvious'' space to use as a local model for a manifold; we can use, e.g., Banach, Hilbert, or Fr\'echet spaces as model spaces, depending on how much structure is needed. In our case, it holds that the space $\I$ is a Fr\'echet manifold. This is Theorem 3.1 of \cite{bauer2014overview}, the proof of which essentially follows by showing that $\I$ is open in $C^{\infty}(\S^1, \R^2)$ and using some general result from \cite{kriegl1997convenient}. Our Definition~\ref{def:tangt-space-of-curves} is then no longer a definition but follows directly from the manifold construction.

Some more work is then needed to construct the quotient space $\mathcal{I}$ which is undertaken in, e.g., \cite{michor2003riemannian} and \cite{cervera1991action}. For example, if we simply use the whole space of immersions, modding out with the diffeomorphism group will not give us a manifold. We need instead to consider only \textit{free} immersion, which are immersions $q$ such that $q \circ \phi = q$ for $\phi \in \mathrm{Diff}(\S^1)$ implies that $\phi = \mathrm{Id}$. Using this, we then get that the quotient space is a manifold (see, e.g., section 2.4 in \cite{michor2003riemannian}).

Even though we know that the quotient space turns out to be a manifold as well it is not obvious that is has the geometric structure as described by our Definition~\ref{def:tang-quotient} and Proposition~\ref{prop:tangent-space-orthogonal}. Fortunately, this holds: For example, section 2.4 in \cite{michor2003riemannian} (which is again based on the general setting of \cite{kriegl1997convenient}) shows that the tangent spaces really do consist of orthogonal vector fields along the shape, and general results about the structure of the quotient space are used to derive our Proposition~\ref{prop:length-quotient} (in particular, this is done on pages 14-15 in \cite{michor2003riemannian}).

Thus, in conclusion, the formal approach agrees with the results obtained through our exposition; of course, with the more formal approach, some properties can be proven instead of defined, which is arguable more satisfactory.




%%% Local Variables:
%%% mode: latex
%%% TeX-master: "mainfile"
%%% reftex-default-bibliography: ("litteratur.bib")
%%% End:
