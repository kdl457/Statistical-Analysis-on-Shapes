\section{Statistical concepts on shape space}
\label{sec:statistical_concepts_on_shape_space}

If we wish to define statistical concepts like mean and variance on our shape space - the manifold of unparametrized curves - we are faced with some challenges. To begin with our manifold fold is infinite-dimensional, and all theory developed so far in this exposition has been on finite-dimensional manifolds. This is an important part of the generalizations of mean and variance, since we can quite easily define a measure on $M$. If a point $x \in M$ can be written in finitely many local coordinates, a measure can be defined via the use of the metric on the induced (finite) basis of $T_x M$. It is not at all obvious how this method of constructing a measure on $M$ generalizes to the infinite-dimensional case.

Secondly, all results regarding existence and uniqueness of mean points have relied on either a global assumption or local assumptions on the Riemannian curvature of the manifold. The behaviour of the curvature of our shape space is not at all well understood, especially since the choice of metric is non-trivial to begin with. One might think that the curvature of $\mathcal{I}$ equipped with the $L^2$-metric is the least difficult to examine, but since the distance function induced by this metric is vanishing, the resulting variance, $\sigma^2_X(y)$ would be $0$ for all random variables $X$ and shapes $y$ (if it is even possible to generalize the variance-formulas in section \ref{sec:mean_and_variance} to the infinite dimensional case.). Thus every point would be a mean point of $X$ which is not a very informative statement.