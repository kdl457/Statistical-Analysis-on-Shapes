\section*{Introduction}
\label{sec:introduction}

A continuously evolving part of statistics is the area of statistical shape analysis where data consists of geometrical shapes instead of observations with a clear numerical representation. In order to perform statistics on geometric shapes, concepts such as distance between shapes has to be well-defined and one has to construct measures on a suitable shape space.

In this project we begin with an introduction of some concepts from Riemannian geometry which are crucial when trying to define statistics on a shape space. This introduction assumes some knowledge of the basic construction of Riemannian manifolds and sets out to define geodesics and the curvature of a manifold $M$ through an abstract construction using the Levi-Civita connection.

The next section introduces generalizations of variance and mean of a $M$-valued random variable when $M$ is a finite-dimensional Riemannian manifold. The mean is generalized through the concept of Riemannian centers of mass and several results are presented regarding the existence and uniqueness of such centers of mass. These results are based on considerations on the curvature of the manifold and the explicit expressions of Riemannian centers of mass is related to the inverse of the exponential map defined through geodesics in the previous section.

Next we introduce the manifolds of parametrized and unparametrized smooth curves in $\R^2$, Imm$(S^1, \R^2)$ and $\mathcal{I}$ respectively. We consider geometric shapes as elements of the latter, and we define metrics on both manifolds. Using these metrics we define the length of a path in the shape space, which can be seen as a deformation of one shape into another. The distance between two shapes is then defined to be the infimum of the length of all paths between the two. We show that a simple metric imposed on $\mathcal{I}$, the $L^2$-metric, induces a vanishing distance function such that the distance between two shapes can be made arbitrarily small.

Since the $L^2$-metric vanishes on $\mathcal{I}$ we turn towards almost local metrics as a way of imposing suitable metrics on $\mathcal{I}$. Almost local metrics are defined and we show that a certain almost local metric separates points in $\mathcal{I}$. The geodesic equations of $\mathcal{I}$ equipped with an almost local metric are not necessarily well-defined, but an example is given of when the distance between two shapes can explicitly be computed.

The last section consists of a discussion on the issues of defining statistical concepts on the shape space. Even if the almost local metrics induce a point-separating distance function on $\mathcal{I}$, there are still challenges with generalizing the constructions from the finite-dimensional case. We reflect upon these challenges and try to point towards further work needed to be done to solve these challenges.








\newpage