
\section{Question / considerations / imprecisions}
\label{sec:quest--cons}

Above we made an intuitively reasonable construction, but ignored some
of technicalities, which we list here.
\begin{itemize}
\item Equivalence classes: We should define the tangent vectors as
  equivalence classes of path in $B$; this should all determine a
  unique vector field.
\item The manifold structure of $B$: We simply used the intuitive idea
  to differentiate in ``time'' for each fixed point on a curve,
  $q_t(x)$. However, as we have not specified the chart for the
  manifold $B$, it is not obvious that this construction corresponds
  to the one made in the finite dimensional case. Technically we
  would need a chart $\phi \colon U \rightarrow F$, with $F$ some
  Fr\'echet space and then show some sort of Fr\'echet-differentiability of the
  composite function $\phi \circ \gamma$.
\item We mentioned that in the finite dimensional case the two
  definitions (through derivates and tangents to curves, respectively)
  are equivalent. It is not obvious that this also hold in the
  infinite dimensional case.
\item At the beginning we eliminated the knowledge of the
  parametrization of a curve $q \in B$ to make the definition of $B$
  simpler. However, we actually use a parametrization later, and thus
  we should make sure that reparametrizations does not matter for the
  construction of the tangent space. (It does not, as it would just
  move the vector field around $\S^1$ according to the
  reparametrization.)
\item Are there any smoothness assumptions (or something) about vector
  field we need to validate? For example, just the fact that the map
  $t \mapsto q_t(x)$ behaves nice does of course \textit{not} imply
  that also the derived vector field $\gamma'_q$ behaves nicely in
  $x$ -- which is what we would need to get a smooth vector field
  (?).
\item Is it correctly formulated that we need to think of $\S^1
  \subset \R^2$ to make sense of a vector field on the circle?
\end{itemize}

\section{Sort of the same construction by the argument from Mumford and Michor}
\label{sec:defin-metr-tang}

\hl{This is the sort of argument they (Mumford and Michor) give in the
  beginning of section 3.2. Not sure how to exactly finish this.}

Consider the following metric on $\I$.

\begin{definition}[The L2 metric on $\I$]
  The \textit{L2 metric} $G^2_c$ at the point $c \in \I$ is defined as
  \begin{equation*}
    G^2_c(h,k) := \int_{\S^{1}} \left\langle{h(\theta)
    , k(\theta)}\right\rangle |c_{\theta}| \diff \theta,
  \end{equation*}
  $h,k \in T_cB = C^{\infty}(\S^1,\R^2)$.
\end{definition}

Adding the parametrization derivative $c_{\theta}$ makes this
invariant under \hl{(orientation preserving?)}
reparametrizations. Because of that, this metric also induces a metric
on the quotient space $B$, which we can identify as follows. Consider
first the tangent space at $c$ to the \textit{orbit} of the diffeomorphism
group
\begin{equation*}
  T_c(c \circ \text{Diff}(\S^{1})), \quad c \circ
  \text{Diff}(\S^{1}) :=
  \left\{
    c \circ \phi \mid \phi \in \text{Diff}(\S^{1})
  \right\} .
\end{equation*}
A path in the orbit $c \circ \text{Diff}(\S^{1})$ reduces to
\begin{equation*}
  t \mapsto
  \left(
    \theta \mapsto (c \circ \phi_t) (\theta)
\right).
\end{equation*}
Differentiating this with respect to $t$ shows that we can identify
\begin{equation}
  \label{eq:tang-orbit}
  T_c(c \circ \text{Diff}(\S^{1})) =
  \left\{
    g c_{\theta} \mid g \in C^{\infty}(\S^1, \R)
  \right\}.
\end{equation}
The \textit{normal space} $\mathcal{N}_c \subset T_c(\I)$ consists of
tangent vectors that are orthogonal to $T_c(c \circ
\text{Diff}(\S^{1}))$ with respect to $G_c^2$. As the projection
\begin{equation*}
  \pi \colon \I \rightarrow B = \I/\text{Diff}(\S^{1})
\end{equation*}
sends the whole orbit to a single point \hl{it follows} that tangents
vectors to the projected element $\pi(c) \in B$ can essentially
be identified with elements of the normal space $\mathcal{N}_c$.
From \eqref{eq:tang-orbit} we see that
\begin{equation*}
  \mathcal{N}_c =
  \left\{
    g i c_{\theta} \mid g \in C^{\infty}(\S^1, \R)
  \right\},
\end{equation*}
with $i c_{\theta} = |c_{\theta}| n_c$, $n_c$ being the unit normal
vector field along $c$. Finally, for a given $h \in T_c(\I)$
we can explicitly find the decomposition of $h$ into the two orthogonal
subspaces of the tangent space by projecting $h$ onto $c_{\theta}$ and
$ic_{\theta}$, which gives
\begin{equation*}
  \begin{aligned}
    h &  = p_{c_{\theta}}(h) + p_{ic_{\theta}}(h)
    \in T_c(c \circ \text{Diff}(\S^{1})) \oplus \mathcal{N}_c
    ,\\
    p_{c_{\theta}}(h) & =
    \frac{\langle h, c_{\theta}\rangle}{|c_{\theta}|^2} c_{\theta}
    \in T_c(c \circ \text{Diff}(\S^{1})), \\
    p_{ic_{\theta}}(h) & =
    \frac{\langle h, ic_{\theta}\rangle}{|c_{\theta}|^2} ic_{\theta}
    \in \mathcal{N}_c,
  \end{aligned}
\end{equation*}
with $p_{v}(u)$ denoting the standard orthogonal projection in $\R^2$.

This allows us to calculate a formula for
the induced metric on $B$. In particular, it allows us to explicitly calculate
the length of a projected path $t \mapsto \pi(q(t,\blank)) \in B $
where $t \mapsto q(t,\blank) \in \I$ is a path in $\I$.
In this setting, at each time point $t$
the tangent vector is $q_t$ so \hl{by the above arguments} we get that
the induced metric is
\begin{equation*}
  \begin{aligned}
    L^2(\pi(q))& = \int_{0}^{1}
    G^2_{q(t,\blank)}(p_{iq_{\theta}}(q_t),p_{iq_{\theta}}(q_t)) \diff t \\
    & = \int_{0}^{1}
    \int_{\S^1}
    \left\langle
      \frac{\langle h, iq_{\theta}\rangle}{|q_{\theta}|^2} iq_{\theta},
      \frac{\langle h, iq_{\theta}\rangle}{|q_{\theta}|^2} iq_{\theta}
  \right\rangle | q_{\theta}|\diff \theta \diff t.
  \end{aligned}
\end{equation*}

\section{Short construction of tangent vectors from alternative definition}
\label{sec:short-constr-tang}

Lee defines the tangent
space at a point $p \in B$ through the notation of
\textit{derivations};

Using this, one can define the notion of a
tanget vector to a path in $B$ passing through the point $p$. Then,
one can define an equivalence relation on the space of such paths and
obtain an equivalent definition of the tanget space, which is more
intuitive. One can also work the other way around and start by
defining the notion of tanget vector to paths in $B$ (as is done on
Wiki). We briefly do that here:

For a neighbourhood $U \subset B$ containing $p$ we have a smooth coordinate
chart $\phi\colon U \rightarrow \R^n$. For a path $\gamma \colon
[-\epsilon,\epsilon] \rightarrow B$ with $\phi(0) = p$, it makes perfect sense to consider
differentiability of the map $\phi \circ \gamma \colon
[0,1]\rightarrow \R^n$. Now, the relation
\begin{equation*}
  \gamma_1 \sim \gamma_2 \iff
  (\phi \circ \gamma_1)'(0) = (\phi \circ \gamma_2)'(0),
\end{equation*}
defines an equivalence relation on the collection of all such paths
$\gamma$. An equivalence class of such paths, denoted by $[\gamma'_p]$ (or simply
$\gamma'_p$), is called a \textit{tangent vector}
at the point $p$. The collection of all tanget vectors make up the
tanget space $T_pB$ at $p$.
