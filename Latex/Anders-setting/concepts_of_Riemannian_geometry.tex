\section*{Concepts of Riemannian geometry}
\label{sec:concepts_of_Riemannian_geometry}

\subsection{Notation}

In the following $M$ denotes a smooth manifold and $TM$ is the tangent bundle of $M$ and $\mathcal{T}(M)$ denotes the space of all vector fields on $M$. For $I \subset \R$, $\gamma: \; I \rightarrow M$ is a curve in $M$, i.e. a smooth map.  

\subsection{Connections}

To consider the geodesic distance between two points in a manifold, geodesics need to be defined in a coordinate-invariant way such that the distance is independent of the coordinate charts. One property of geodesics in a Euclidean space, straight lines, is that they have acceleration $0$. In order to make sense of acceleration of a curve in a manifold, we need to be able to compute "differences" between tangent spaces along the curve. \textit{Connections} are exactly a way of making computations between tangent spaces possible - they allow us to differentiate vector fields along curves.\\[0.2 cm]
Since our use of connections is to define geodesics, we define connections in the tangent bundle of a manifold (instead of defining them generally on smooth sections of vector bundles) follwing chapter $4$ of \hl{RiemannLee}. 

\begin{definition}
A connection in $TM$ is a map
\begin{align*}
\nabla \, : \, \mathcal{T}(M) \times \mathcal{T}(M) \rightarrow \mathcal{T}(M),
\end{align*}
written $(X, Y) \mapsto \nabla_X Y$ satisfying (for $f$, $g \in C^\infty(M)$ and $a$, $b \in \R$);
\begin{align*}
& a) \; \nabla_{fX_1 + gX_2} Y = f\nabla_{X_1} Y + g\nabla_{X_2} Y \; \; \; \; & \text{(linearity over} \; C^\infty(M) \, \text{in} \, X) \\
& b) \; \nabla_X (aY_1 + bY_2) = a\nabla_X Y_1 + b\nabla_X Y_2 & \text{(linearity over} \; \R \, \text{in} \, Y) \\
& c) \; \nabla_X (fY) = f\nabla_X Y + (Xf)Y & \text{(product rule)}
\end{align*}
\end{definition}

In accordance with connections allowing "differences" between tangent spaces, $ \nabla_X Y $ is called the $\textit{covariant derivative of Y in the direction of X}$. Note that the product rule for connections is identical to the product rule of derivations. To use connections to derivate along curves, we need the definition of a \textit{vector field along a curve}, which is a smooth map $V: \, I \rightarrow TM$ such that $V(t) \in T_{\gamma(t} M$ for all $t \in I$. The prime example of a vector field along a curve is its velocity, $\dot{\gamma}(t) \in T_{\gamma(t} M$, which acts on functions, $f \in C^\infty(M)$, by
\begin{align*}
\dot{\gamma}(t) f = \frac{\text{d}}{\text{dt}} (f \circ \gamma)(t).
\end{align*} 
We denote by $\mathcal{T}(\gamma)$ all vector fields along $\gamma$. 
To define geodesics all we now need is to define what is means to take the covariant derivative of $V \in \mathcal{T}(\gamma)$ along $\gamma$. This covariant derivative is noted $D_t V$ and is has the following properties.

\begin{lemma}
Let $\nabla$ be a linear connection on $M$. For each $\gamma: \; I \rightarrow M$, $\nabla$ determines a unique operator
\begin{align*}
D_t: \; \mathcal{T}(\gamma) \rightarrow \mathcal{T}(\gamma),
\end{align*} 
satisfying (for $f$, $g \in C^\infty(I)$ and $a$, $b \in \R$);
\begin{align*}
& a) \; D_t(aV + bW) = aD_tV + bD_tW \; \; \; \; & \text{(linearity over} \; \R \\
& b) \; D_t(fV) = \dot{f}V + fD_tV & \text{(product rule)} \\
& c) \; \text{If V is extendible, then for any extension} \, \tilde{V} \, \text{of V}, \; \; D_tV(t) = \nabla_{\dot{\gamma}(t)} \tilde{V}.
\end{align*}
\end{lemma}
\begin{proof}
Proof of Lemma 4.9 in \hl{RiemannLee}
\end{proof}

$V$ is said to be extendible if it can be constructed by any vector field on $M$, $\tilde{V}$ by letting $V(t) := \tilde{V}_{\gamma(t)}$. This is not always the case; if $V$ is the velocity of an intersecting curve $\gamma$ with different covariant derivative at the intersection times. The covariant derivative of the velocity of a curve is now used to define a geodesic.

\begin{definition}
Let $\nabla$ be a linear connection on $M$ and $\gamma$ a curve in $M$. The acceleration of $\gamma$ is $D_t \dot{\gamma}(t)$. If this vector field is zero, $D_t \dot{\gamma}(t) \equiv 0$, then $\gamma(t)$ is a geodesic with respect to $\nabla$
\end{definition}

It follows from Theorem $4.10$ in \hl{RiemannLee} that for any manifold, $M$, with a linear connection, for any $p \in M$ and $V \in T_pM$ and $t_0 \in \R$ there exists an un-extendable geodesic, $\gamma_V: \; I \rightarrow M$, with $\gamma(0) = p$ and $\dot{\gamma}(0) = V$. The geodseic is called the (maximal) geodesic with initial value $p$ and initial velocity $V$. \\[0.2 cm]

In this construction of geodesics the only necessary structure of $M$ is that it should be a smooth manifold. When $M$ is also equipped with a Riemannian metric, making $M$ a Riemannian manifold, the choice of connection (determining the geodesics) should in some way respect the metric. Geodesics resulting from this specific choice of connection are called \textit{Riemannian geodesics}.

\subsection{Riemannian Geodesics and the Exponential Map}

Let $M$ be Riemannian manifold with metric $g$. To define Riemannian geodesics, we must first choose a specific connection on $M$ with two properties - \textit{compatability w.r.t. $g$} and \textit{symmetric} (these properties arise when trying to generalize the tangential conenction of a manifold submersed in $\R^n$).

\begin{definition}
A conneciton on $M$ is \textit{compatible with g} if the product rule
\begin{align*}
\nabla_X (Y, Z) = \langle \nabla_X Y, Z \rangle + \langle Y, \nabla_X Z \rangle
\end{align*}
holds for all vector fields, $X, Y$ and $Z$.
\end{definition}

By Lemma $5.2$ in \hl{RiemannLee} this condiiton is equivalnt with
\begin{align*}
\frac{d}{dt} \langle V, W \rangle = \langle D_t V, W \rangle + \langle V, D_t W \rangle,
\end{align*} 
for $V, W$ being vector fields along any curve $\gamma$. \\[0.2 cm]

The defintion of a symmetric connection involves the Lie bracket of two vector fields. If we think of vector fields on $M$, $X, Y$, as derivtations acting on $C^\infty (M)$, then the Lie brakcet of $X$ and $Y$, $[X , Y]$, is the vector field (derivation) which acts on $f \in C^\infty (M)$ by

\begin{align*}
[X , Y] (f) := X(Y(f)) - Y(X(f)),
\end{align*}
where $X(f) \in C^\infty (M)$ is the function which evaluated at $p$ is the derivative of $f$ at $p$ in the direction of $X(p)$. 

\begin{definition}
A connection on $M$ is \textit{symmetric} if 
\begin{align*}
\nabla_X Y - \nabla_Y X \equiv [X, Y].
\end{align*}
\end{definition}

(Note that interchanging $X$ and $Y$ makes sense, since $[X, Y] = - [Y, X]$.) A symmetric connection is also called \textit{torsion free}, which corresponds to the vector fields along any curve not being "twisted" when they are parallel transported along the curve. By the Fundamental Theorem of Riemannian Geometry, given any Riemannian manifold $(M,g)$, there exists a unique connection, $\nabla$, on $M$ that is symmetric and compatible with $M$. This connection is called the Riemannian connection of the Levi-Civita connection, and geodesics with respect to this connection are called Riemannian geodesics. Since this choice of connection is unique, geodesics in a Riemannian manifold will always mean Riemannian geodesics. Such geodesics can be used to define the exponential map, $\exp: \; TM \rightarrow M$.  

\begin{definition}
Given a point $p$ in a Riemannian manifold, $M$, and a vector $v \in T_p M$, the \textit{exponential map} is defined by $\exp_p (v) = \gamma_v(1)$ where $\gamma$ is the unique geodesic with $\gamma(0) = p$ and $\dot{\gamma}(0) = v$.
\end{definition}

The exponential map pushes a point $p \in M$ a unit distance in the direction of $v$ along the geodesic $\gamma$. Since $\dot{\gamma}(0) = v$, the differential of $\exp_p$ at $p$ is $v$, and so $\exp_p$ is a local diffeomorphism by the implicit function theorem. A Riemannian manifold is said to be \textit{geodesically complete} if every un-exentedable geodesic is defined on all of $\R$, and it follows from the Hopf-Rinow theorem that a manifold is geodesically complete if and only if, there exists a $p \in M$ such that $\exp$ is defined on all of $T_p M$.\\[0.2 cm] 
If the geodesic $\gamma_v$ is defined on all of $\R$, one can investigate for which values of $t$, the extended geodesic, $\gamma_v(t) = \exp_p (t v)$ is still a geodesic from $p$ to $\exp_p (t v)$. If $\gamma_v$ is a geodesic up to time $t_0$ and not after $t_0$, then $t_0$ is called a \textit{cut point}. The set of all cut points of all geodesics starting at $p$ is called the \textit{cut locus} and is denoted $C(p)$.

\subsection{Curvature}

When defining statistical propoerties such as mean and variance on Riemannian manifolds, measuring geometric propoerties of the manifold is needed. It is therefore of interest to determine on which manifolds these measurements are identical - that is to determine which manifolds are (locally) isomorphic. One way of determining this is to find a local invariant property of a manifold which is preserved by isometries (such that measurements of length are preserved). This property will be the \textit{curvature}. \\[0.2 cm]

Let $M = \R^n$ equipped with the Euclidean metric and consider a vector field $Z$. Given two other vector fields, $X$ and $Y$, we can now differentiate $Z$ first along $X$ and then along $Y$ by using the Riemannian connection;
\begin{align*}
\nabla_Y \nabla_X Z,
\end{align*}
and in the opposite order by $\nabla_X \nabla_Y Z$. If $R^n = R^2$ and $X$ and $Y$ where just vector fields corresponding to local coordinates then, by commutativity of second order derivatives,
\begin{align*}
\nabla_Y \nabla_X Z - \nabla_X \nabla_Y Z = \nabla_Y (\partial_1 Z^k \partial_k) - \nabla_X (\partial_2 Z^k \partial_k) = \partial_2 \partial_1 Z^k \partial_k - \partial_1 \partial_2 Z^k \partial_k = 0.
\end{align*}
But if $X$ and $Y$ are arbitrary vector fields, this identity does not necessarily hold, since
\begin{align*}
\nabla_Y \nabla_X Z - \nabla_X \nabla_Y Z = XYZ^k \partial_k - YXZ^k \partial_k = (XYZ^k - YXZ^k)\partial_k,
\end{align*}
where $XYZ^k = X \left( Z \left( Z^k \right) \right)$ to ease notation. The action of the Lie bracket of $X$ and $Y$ is recognized in the parenthesis, and thus for $\R^n$ the following identity holds;
\begin{align}\label{flat}
\nabla_Y \nabla_X Z - \nabla_X \nabla_Y Z = \nabla_{[X,Y]} Z.
\end{align}
Since this identity depends on the Levi-Civita connection, it holds for all manifolds isometric to $\R^n$, and manifolds for which \ref{flat} holds will be called \textit{flat} manifolds. To define the curvature of a manifold is then to determine how "un-flat" the manifold is, by considering the \textit{curvature transformation}, $R: \; \mathcal{T}(M) \times \mathcal{T}(M) \times \mathcal{T}(M) \rightarrow \mathcal{T}(M)$, defined by
\begin{align*}
R(X,Y)Z =  \nabla_Y \nabla_X Z - \nabla_X \nabla_Y Z - \nabla_{[X,Y]} Z,
\end{align*}
which is identically zero for flat manifolds. The curvature transformation can then be used to determine the curvature of a vector field by defining the \textit{curvature tensor};

\begin{definition}
The curvature on a Riemannian manifold is
\begin{align*}
Rm(X,Y,Z,W) = \langle R(X,Y)Z, W \rangle,
\end{align*}
with $\langle \cdot ,  \cdot \rangle $ being the inner product determined by the Riemannian metric.
\end{definition}


