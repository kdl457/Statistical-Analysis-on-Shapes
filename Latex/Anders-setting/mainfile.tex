\documentclass[a4,danish]{article}
\documentclass[11pt,a4paper,twoside,openany,final]{memoir}
\usepackage[utf8]{inputenc}
\usepackage[twoside]{geometry}
%\usepackage[T1]{fontenc}
\usepackage[english]{babel}
\usepackage{amsmath}
\usepackage{amsfonts}
\usepackage{amsthm}
\usepackage[usenames,dvipsnames]{xcolor}
\usepackage{tikz}
\usepackage{amssymb}
\usepackage{graphicx}
\usepackage{flexisym}
\usepackage{hyperref}
\usepackage{xr}
\usepackage[all]{xy}
\usepackage{tikz-cd}
\usepackage{tkz-graph} % To make graphs
\usetikzlibrary{arrows}
\usepackage{tkz-tab}
\usepackage{hyperref}
\usepackage[style=authoryear,backend=bibtex,natbib]{biblatex}
\usepackage{filecontents}
\usepackage[english, status=draft]{fixme}
\fxusetheme{color}
\usepackage{cleveref} 
\usepackage[backgroundcolor=cyan]{todonotes}
\usepackage{wallpaper}
\usepackage{faktor}
\usepackage{nicefrac}
%\usepackage{txfonts}
\usepackage{afterpage} %Til at kunne indsætte tomme sider
\usepackage{pgfplots} %Til at kunne lave grafer
\usepackage{emptypage} %Fjerner sidetal og sidehoveder på tomme sider
\usepackage{multirow}
\usepackage{listings} % To use R language
\lstset{literate=%
{æ}{{\ae}}1
{å}{{\aa}}1
{ø}{{\o}}1
{Æ}{{\AE}}1
{Å}{{\AA}}1
{Ø}{{\O}}1
}
\lstset{ 
  language=R,                     % the language of the code
  basicstyle=\footnotesize\ttfamily, % the size of the fonts that are used for the code
  numberstyle=\tiny\color{blue},  % the style that is used for the line-numbers
  stepnumber=1,                   % the step between two line-numbers. If it is 1, each linewill be numbered
  numbersep=5pt,                  % how far the line-numbers are from the code
  backgroundcolor=\color{white},  % choose the background color. You must add \usepackage{color}
  showspaces=false,               % show spaces adding particular underscores
  showstringspaces=false,         % underline spaces within strings
  showtabs=false,                 % show tabs within strings adding particular underscores
%  frame=single,                   % adds a frame around the code
  rulecolor=\color{black},        % if not set, the frame-color may be changed on line-breaks within not-black text (e.g. commens (green here))
  tabsize=2,                      % sets default tabsize to 2 spaces
  captionpos=b,                   % sets the caption-position to bottom
  breaklines=true,                % sets automatic line breaking
  breakatwhitespace=false,        % sets if automatic breaks should only happen at whitespace
  keywordstyle=\color{RoyalBlue},      % keyword style
  commentstyle=\color{ForestGreen},   % comment style
  xleftmargin=0.5cm,
  xrightmargin=0.5cm
}


%\DisemulatePackage{setspace} START ALTERNATIV SPACING + MARGIN %
%\usepackage{setspace} %
%\doublespacing %
%\setlrmarginsandblock{5cm}{3cm}{*}
%\setulmarginsandblock{2.5cm}{2.5cm}{*}
%\checkandfixthelayout SLUT ALTERNATIV SPACING + MARGIN

\pgfplotsset{width=10cm,compat=1.9}

\begin{filecontents}{bibtest.bib}
@book{Clarke,
  title={Functional Analysis, Calculus of Variations and Optimal Control},
  author={Francis Clarke},
  volume={1},
  year={2013},
  publisher={Springer-Verlag London}
}
@book{Hansen,
  title={Measure Theory},
  author={Ernst Hansen},
  volume={4},
  year={2009},
  publisher={Department of Mathematical Sciences, University of Copenhagen}
}
@article{Topsoee,
  title={Compactness in spaces of measures},
  author={Flemming Topsøe},
  journal={Studia Mathematica},
  volume={36},
  number={3},
  year={1970},
  pages={195--212}
}
@book{Topsoee2,
  title={Topology and measure},
  author={Flemming Topsøe},
  volume={1},
  year={1970},
  publisher={Springer-Verlag, Berlin}
}
@article{Strassen,
  title={The existence of probability measures with given marginals},
  author={Strassen, Volker},
  journal={The Annals of Mathematical Statistics},
  volume={36},
  number={2},
  year={1965},
  pages={423--439}
}
@book{Hoffmann,
  title={Probability in Banach space},
  author={Jørgen Hoffmann-Jørgensen},
  volume={},
  year={1977},
  publisher={Springer}
}
@book{Rudin,
  title={Real and complex analysis},
  author={Walter Rudin},
  volume={3},
  year={1987},
  publisher={McGraw-Hill, Inc.}
}
@book{Rudin2,
  title={Functional analysis},
  author={Walter Rudin},
  volume={2},
  year={1991},
  publisher={McGraw-Hill, Inc.}
}
@book{Aliprantis,
  title={Infinite Dimensional Analysis: A Hitchhiker's Guide},
  author={Charalambos D. Aliprantis and Kim C. Border},
  volume={3},
  year={2006},
  publisher={Springer-Verlag Berlin}
}
@article{Lindvall,
  title={On Strassen's Theorem on stochastic domination},
  author={Torgny Lindvall},
  journal={Electronic Communications in Probability},
  volume={4},
  number={7},
  year={1999},
  pages={51--59}
}
@book{Billingsley,
  title={Convergence of Probability Measures},
  author={Patrick Billingsley},
  volume={2},
  year={2013},
  publisher={John Wiley and Sons}
}
@article{Kamae,
  title={Stochastic inequalities on partially ordered spaces},
  author={Teturo Kamae and Ulrich Krengel and George L. O'Brien},
  journal={The Annals of Probability},
  volume={5},
  number={6},
  year={1977},
  pages={899--912}
}
@book{Sokol,
  title={Advanced Probability},
  author={Alexander Sokol and Anders Rønn-Nielsen},
  volume={4},
  year={2016},
  publisher={Department of Mathematical Sciences, University of Copenhagen}
}
@book{Bladt,
  title={Matrix–exponential distributions in Applied Probability},
  author={Mogens Bladt and Bo Friis Nielsen},
  volume={1},
  year={2017},
  publisher={Springer (Pending)}
}
@article{Rosenblum,
  title={Simple Examples of Estimating Causal Effects Using Targeted Maximum Likelihood Estimation},
  author={Michael Rosenblum and Mark J. van der Laan},
  journal={U.C. Berkeley Division of Biostatistics Working Paper Series},
  volume={Working Paper 262},
  year={2010}
}
@Inbook{Kennedy,
author="Kennedy, Edward H.",
editor="He, Hua
and Wu, Pan
and Chen, Ding-Geng (Din)",
title="Semiparametric Theory and Empirical Processes in Causal Inference",
bookTitle="Statistical Causal Inferences and Their Applications in Public Health Research",
year="2016",
publisher="Springer International Publishing",
address="Cham",
pages="141--167",
isbn="978-3-319-41259-7",
doi="10.1007/978-3-319-41259-7_8",
url="https://doi.org/10.1007/978-3-319-41259-7_8"
}
@book{TargetedLearning,
  title={Targeted Learning in Data Science},
  author={Mark J. van der Laan and Sherri Rose},
  volume={1},
  year={2011},
  publisher={Springer}
}
@article{Wikkelsoee,
  title={Prediction of postpartum blood transfusion -- risk factors and recurrence.},
  author={AJ Wikkelsøe and S Hjortøe and TA Gerds and AM Møller and J Langhoff-Roos}, 
  journal={The journal of maternal-fetal and neonatal medicine},
  volume={27},
  number={16},
  year={2014}
}
@book{Causality,
  title={Elements of causal inference},
  author={Jonas Peters and Dominik Janzing and Bernhard Schölkopf},
  volume={1},
  year={2017},
  publisher={MIT Press}
}
@book{Asymptotic,
  title={Asymptotic Statistics},
  author={A. W. Van der Vaart},
  volume={3},
  year={2000},
  publisher={Cambridge University Press}
}
@article{PPHcause,
  title={Recent Advances in the Management of Major Postpartum Haemorrhage - A Review},
  author={P Reddi Rani1 and Jasmina Begum}, 
  journal={Journal of Clinical and Diagnostic Research},
  volume={11},
  number={2},
  year={2017}
}
@Manual{SuperLearner,
    title = {SuperLearner: Super Learner Prediction},
    author = {Eric Polley and Erin LeDell and Chris Kennedy and Mark {van der Laan}},
    year = {2018},
    note = {R package version 2.0-23},
    url = {https://CRAN.R-project.org/package=SuperLearner}
}
@Article{tmle,
    title = {{tmle}: An {R} Package for  Targeted Maximum Likelihood Estimation},
    author = {Susan Gruber and Mark J. {van der Laan}},
    journal = {Journal of Statistical Software},
    year = {2012},
    volume = {51},
    number = {13},
    pages = {1--35},
    url = {http://www.jstatsoft.org/v51/i13/},
}
@book{RiemannLee,
  title={Riemannian Manifolds},
  author={John M Lee},
  volume={1},
  year={1997},
  publisher={Springer}
}



\end{filecontents}

\addbibresource{bibtest.bib}


\chapterstyle{verville}


\setlength{\parindent}{0em}
\setlength{\parskip}{1em}
\renewcommand{\baselinestretch}{1}


\DeclareMathOperator{\supp}{supp}
\DeclareMathOperator{\Ext}{Ext}
\DeclareMathOperator{\Aut}{Aut}
\DeclareMathOperator{\Ran}{Ran}
\DeclareMathOperator{\Prob}{Prob}
\DeclareMathOperator{\conv}{conv}
\DeclareMathOperator{\AR}{AR}
\DeclareMathOperator{\Homeo}{Homeo}

\makepagestyle{abs}
    \makeevenhead{abs}{}{}{}
    \makeoddhead{abs}{}{}{}
    \makeevenfoot{abs}{}{\scshape I }{}
    \makeoddfoot{abs}{}{\scshape  I }{}
    %\makeheadrule{abs}{\textwidth}{\normalrulethickness}
    %\makefootrule{abs}{\textwidth}{\normalrulethickness}{\footruleskip}
\pagestyle{abs}


\makepagestyle{cont}
    \makeevenhead{cont}{}{}{}
    \makeoddhead{cont}{}{}{}
    \makeevenfoot{cont}{}{\scshape II }{}
    \makeoddfoot{cont}{}{\scshape  II }{}
    %\makeheadrule{abs}{\textwidth}{\normalrulethickness}
    %\makefootrule{abs}{\textwidth}{\normalrulethickness}{\footruleskip}
\pagestyle{cont}

\newcommand{\lv}{\lVert}
\newcommand{\rv}{\rVert}


\renewcommand\chaptermarksn[1]{}
\nouppercaseheads
\createmark{chapter}{left}{shownumber}{}{.\space}
\makepagestyle{dut}
    \makeevenhead{dut}{\scshape\rightmark}{}{}
    \makeoddhead{dut}{\scshape\leftmark}{}{}
    \makeevenfoot{dut}{}{\scshape $-$ \thepage\ $-$}{}
    \makeoddfoot{dut}{}{\scshape $-$ \thepage\ $-$}{}
    \makeheadrule{dut}{\textwidth}{\normalrulethickness}
    \makefootrule{dut}{\textwidth}{\normalrulethickness}{\footruleskip}
\pagestyle{dut}

\makepagestyle{chap}
    \makeevenhead{chap}{}{}{}
    \makeoddhead{chap}{}{}{}
    \makeevenfoot{chap}{}{\scshape $-$ \thepage\ $-$}{}
    \makeoddfoot{chap}{}{\scshape $-$ \thepage\ $-$}{}
    \makefootrule{chap}{\textwidth}{\normalrulethickness}{\footruleskip}
\copypagestyle{plain}{chap}

\newcommand{\R}{\mathbb{R}}
\newcommand{\C}{\mathbb{C}}
\newcommand{\N}{\mathbb{N}}
\newcommand{\E}{\mathrm{E}}
\newcommand{\Var}{\mathrm{Var}}
\newcommand{\mbr}{(X,\mathcal{A})}
\newcommand{\Z}{\mathbb{Z}}
\newcommand{\Q}{\mathbb{Q}}
\newcommand{\F}{\mathcal{F}}
\newcommand{\A}{\mathcal{A}}
\newcommand{\cc}{C_c}
\newcommand{\PP}{\mathcal{P}}
\newcommand{\B}{\mathcal{B}}
\newcommand{\ee}{\epsilon}
\newcommand{\la}{\lambda}
\renewcommand{\H}{\mathcal{H}}
\newcommand{\pp}{\text{Prob}}
\newcommand{\U}{\mathcal{U}}
\newcommand{\dd}{\mathrm{d}}

\newcommand{\nocontentsline}[3]{}
\newcommand{\tocless}[2]{\bgroup\let\addcontentsline=\nocontentsline#1{#2}\egroup} % Fjern sections fra Table of Contents

\makeatletter
\newcommand{\Spvek}[2][r]{%
  \gdef\@VORNE{1}
  \left(\hskip-\arraycolsep%
    \begin{array}{#1}\vekSp@lten{#2}\end{array}%
  \hskip-\arraycolsep\right)}

\def\vekSp@lten#1{\xvekSp@lten#1;vekL@stLine;}
\def\vekL@stLine{vekL@stLine}
\def\xvekSp@lten#1;{\def\temp{#1}%
  \ifx\temp\vekL@stLine
  \else
    \ifnum\@VORNE=1\gdef\@VORNE{0}
    \else\@arraycr\fi%
    #1%
    \expandafter\xvekSp@lten
  \fi}
\makeatother

\def\acts{\curvearrowright}

\newcommand{\K}{\mathbb{K}}

\newtheoremstyle{break}
	{\topsep}{\topsep}
	{\bfseries}{}
	{\newline}{}
\theoremstyle{break}
\newtheorem{theorem}[subsection]{Theorem}
\newtheorem{lemma}[subsection]{Lemma}
\newtheorem{proposition}[subsection]{Proposition}
\newtheorem{corollary}[subsection]{Corollary}
\newtheorem{definition}[subsection]{Definition}
\newtheoremstyle{Break}
	{\topsep}{\topsep}
	{}{}
	{\bfseries}{}
	{\newline}{}
\theoremstyle{Break}
\newtheorem{example}[subsection]{Example}
\newtheorem{remark}[subsection]{Remark}
\newtheorem{note}[subsection]{Note}
\setcounter{secnumdepth}{0}
\usepackage{xpatch}
\xpatchcmd{\proof}{\ignorespaces}{\mbox{}\\\ignorespaces}{}{}

\newcommand*{\diff}{\mathop{}\!\mathrm{d}}

\newcommand\blankpage{%
    \null
    \thispagestyle{empty}%
    \addtocounter{page}{-1}%
    \newpage}
    
\DeclareNameAlias{sortname}{last-first} % Med flere forfattere bliver alle navne på formen
\DeclareNameAlias{default}{last-first} % : "LastName, FirstName MiddleName

\newcommand{\blank}{\makebox[1ex]{\textbf{$\cdot$}}} % Command to placeholder

\newcommand{\indep}{\rotatebox[origin=c]{90}{$\models$}} % Independent symbol

\clearpage
\thispagestyle{empty}

\begin{titlingpage}
	\ThisLRCornerWallPaper{1}{frontpage/2.pdf}	
	\vspace*{5.5cm}
	\noindent
	{\large\textsc{Anders Munch and Mads Møller Pedersen}}\\[0.5cm]
	{\large\textsc{Analysis of shape spaces}}\\[0.1cm]
	\vfill\noindent
	{\large\textsc{Project outside of the course scope}}\\[0.2cm]
	\noindent
	{\large\textsc{Department of Mathematical Sciences}}\\[0.2cm]
	\noindent
	{\large\textsc{University of Copenhagen}}\\[1cm]
	{\large\textsc{Advisor \\[0.2cm] {\Large Stefan Sommer }}}\\[1cm]
	{\large\textsc{November 9, 2018}}
	\let\cleardoublepage\clearpage
\end{titlingpage}
\normalfont
\restoregeometry
\cleardoublepage

\begin{document}
\maketitle
\newpage


\begin{abstract}
\noindent
Performing statistics on data consisting of shapes rather than numerical values requires geometric constructions on the manifold of shapes. In this project we develop concepts of Riemannian geometry which are used to define the variance and mean of a random variable with values in a finite-dimensional manifold. We construct the infinite-dimensional manifolds of parametrized and un-parametrized smooth curves in $\R^2$ called shape spaces. We equip the shape spaces with different metrics some of which can be used to determine the distance between two shapes. Finally we discuss if the theory of statistics on finite-dimensional manifolds can be generalized to the infinite-dimensional shape spaces.

\end{abstract}
\newpage
\tableofcontents
\newpage


\section*{Introduction}
\label{sec:introduction}

A continuously evolving part of statistics is the area of statistical shape analysis where data consists of geometrical shapes instead of observations with a clear numerical representation. In order to perform statistics on geometric shapes, concepts such as distance between shapes has to be well-defined and one has to construct measures on a suitable shape space.

In this project we begin with an introduction of some concepts from Riemannian geometry which are crucial when trying to define statistics on a shape space. This introduction assumes some knowledge of the basic construction of Riemannian manifolds and sets out to define geodesics and the curvature of a manifold $M$ through an abstract construction using the Levi-Civita connection.

The next section introduces generalizations of variance and mean of a $M$-valued random variable when $M$ is a finite-dimensional Riemannian manifold. The mean is generalized through the concept of Riemannian centers of mass and several results are presented regarding the existence and uniqueness of such centers of mass. These results are based on considerations on the curvature of the manifold and the explicit expressions of Riemannian centers of mass is related to the inverse of the exponential map defined through geodesics in the previous section.

Next we introduce the manifolds of parametrized and unparametrized smooth curves in $\R^2$, Imm$(S^1, \R^2)$ and $\mathcal{I}$ respectively. We consider geometric shapes as elements of the latter, and we define metrics on both manifolds. Using these metrics we define the length of a path in the shape space, which can be seen as a deformation of one shape into another. The distance between two shapes is then defined to be the infimum of the length of all paths between the two. We show that a simple metric imposed on $\mathcal{I}$, the $L^2$-metric, induces a vanishing distance function such that the distance between two shapes can be made arbitrarily small.

Since the $L^2$-metric vanishes on $\mathcal{I}$ we turn towards almost local metrics as a way of imposing suitable metrics on $\mathcal{I}$. Almost local metrics are defined and we show that a certain almost local metric separates points in $\mathcal{I}$. The geodesic equations of $\mathcal{I}$ equipped with an almost local metric are not necessarily well-defined, but an example is given of when the distance between two shapes can explicitly be computed.

The last section consists of a discussion on the issues of defining statistical concepts on the shape space. Even if the almost local metrics induce a point-separating distance function on $\mathcal{I}$, there are still challenges with generalizing the constructions from the finite-dimensional case. We reflect upon these challenges and try to point towards further work needed to be done to solve these challenges.








\newpage
\newpage
\section{Concepts of Riemannian geometry}
\label{sec:concepts_of_Riemannian_geometry}

We start by introducing some important concepts of Riemannian geometry. These will be used to define ideas like distance and length on manifolds which are crucial in attempting to perform statistics on manifolds. We define geodesics which are used to define the distance between two points, and the concept of curvature, which plays an important role in existence of mean points, is introduced. In the following $p \in M$ denotes a point in a smooth manifold and $TM$ the tangent bundle of $M$ and $\mathcal{T}(M)$ denotes the space of all vector fields on $M$. For $I \subset \R$, $\gamma: \; I \rightarrow M$ is a curve in $M$, i.e. a smooth map, and we denote by $\gamma_t$ the time-derivative of the curve.

\subsection{Connections}

To consider the shortest (or geodesic) distance between two points in a manifold, geodesics need to be defined in a coordinate-invariant way such that the distance is independent of the coordinate charts. One property of geodesics in a Euclidean space, straight lines, is that they have acceleration $0$. In order to make sense of acceleration of a curve in a manifold, we need to be able to compute "differences" between tangent spaces along the curve. \textit{Connections} are exactly a way of making computations between tangent spaces possible - they allow us to differentiate vector fields along curves.

Since our use of connections is to define geodesics, we define connections in the tangent bundle of a manifold (instead of defining them generally on smooth sections of vector bundles) following chapter $4$ of \cite{lee2006riemannian}.

\begin{definition}
A connection in $TM$ is a map
\begin{align*}
\nabla \, : \, \mathcal{T}(M) \times \mathcal{T}(M) \rightarrow \mathcal{T}(M),
\end{align*}
written $(X, Y) \mapsto \nabla_X Y$ satisfying (for $f$, $g \in C^\infty(M)$ and $a$, $b \in \R$);
\begin{align*}
& a) \; \nabla_{fX_1 + gX_2} Y = f\nabla_{X_1} Y + g\nabla_{X_2} Y \; \; \; \; & \text{(linearity over} \; C^\infty(M) \, \text{in} \, X) \\
& b) \; \nabla_X (aY_1 + bY_2) = a\nabla_X Y_1 + b\nabla_X Y_2 & \text{(linearity over} \; \R \, \text{in} \, Y) \\
& c) \; \nabla_X (fY) = f\nabla_X Y + (Xf)Y & \text{(product rule)}
\end{align*}
\end{definition}

In accordance with connections allowing "differences" between tangent spaces, $ \nabla_X Y $ is called the $\textit{covariant derivative of Y in the direction of X}$. Note that the product rule for connections is identical to the product rule of derivations on a manifold. To use connections to derivate along curves, we need the definition of a \textit{vector field along a curve}, which is a smooth map $V: \, I \rightarrow TM$ such that $V(t) \in T_{\gamma(t)} M$ for all $t \in I$. The prime example of a vector field along a curve is its velocity, $\dot{\gamma}(t) \in T_{\gamma(t)} M$, which acts on functions, $f \in C^\infty(M)$, by
\begin{align*}
\dot{\gamma}(t) f = \frac{\text{d}}{\text{dt}} (f \circ \gamma)(t).
\end{align*}
We denote by $\mathcal{T}(\gamma)$ all vector fields along $\gamma$.
To define geodesics all we now need is to define what is means to take the covariant derivative of $V \in \mathcal{T}(\gamma)$ along $\gamma$. This covariant derivative is noted $D_t V$ and is has the following properties.

\begin{lemma}
Let $\nabla$ be a connection on $M$. For each $\gamma: \; I \rightarrow M$, $\nabla$ determines a unique operator
\begin{align*}
D_t: \; \mathcal{T}(\gamma) \rightarrow \mathcal{T}(\gamma),
\end{align*}
satisfying (for $f$, $g \in C^\infty(I)$ and $a$, $b \in \R$);
\begin{align*}
& a) \; D_t(aV + bW) = aD_tV + bD_tW & \text{(linearity over} \; \R \\
& b) \; D_t(fV) = \dot{f}V + fD_tV & \text{(product rule)} \\
& c) \; \text{If V is extendible, then for any extension} \, \tilde{V} \, \text{of V}, \; \; & D_tV(t) = \nabla_{\dot{\gamma}(t)} \tilde{V}.
\end{align*}
\end{lemma}
\begin{proof}
Proof of Lemma 4.9 in \cite{lee2006riemannian}
\end{proof}

$V$ is said to be extendible if it can be constructed by a vector field on $M$, $\tilde{V}$ by letting $V(t) := \tilde{V}_{\gamma(t)}$. This is not always the case; for example if $V$ is the velocity of an intersecting curve $\gamma$ with different covariant derivative at the intersection times. The covariant derivative of the velocity of a curve is now used to define a geodesic.

\begin{definition}
Let $\nabla$ be a linear connection on $M$ and $\gamma$ a curve in $M$. The acceleration of $\gamma$ is $D_t \dot{\gamma}(t)$. If this vector field is zero, $D_t \dot{\gamma}(t) \equiv 0$, then $\gamma(t)$ is a geodesic with respect to $\nabla$
\end{definition}

One can show (see for instance Theorem $4.10$ in \cite{lee2006riemannian}) that for any manifold with a connection, for any $p \in M$ and $v \in T_pM$ and $t_0 \in \R$ there exists an un-extendable geodesic, $\gamma_v: \; I \rightarrow M$, with $\gamma_v(0) = p$ and $\dot{\gamma}_v(0) = V$ (un-extendable in the sense that no geodesic exists defined on a larger interval than $I$). The geodseic is called the (maximal) geodesic with initial value $p$ and initial velocity $v$.

In this construction of geodesics the only necessary structure of $M$ is that it should be a smooth manifold. When $M$ is also equipped with a Riemannian metric, making $M$ a Riemannian manifold, the choice of connection (determining the geodesics) should in some way respect the metric. Geodesics resulting from this specific choice of connection are called \textit{Riemannian geodesics}.

\subsection{Riemannian Geodesics and the Exponential Map}

Let $M$ be Riemannian manifold with metric $g$. The metric is defined on $TM$ and we write $g(X,Y) = \langle X, Y \rangle_g$ for $X, Y \in TM$. When the choice of metric is clear, the subscript of the inner product is droppd. To define Riemannian geodesics, we must first choose a specific connection on $M$ with two properties - \textit{compatability w.r.t. $g$} and \textit{symmetry} (these properties arise when trying to generalize the so-called \textit{tangential connection} of a manifold submersed in $\R^n$).

\begin{definition}
A connection on $M$ is \textit{compatible with g} if the product rule
\begin{align*}
\nabla_X (Y, Z) = \langle \nabla_X Y, Z \rangle_g + \langle Y, \nabla_X Z \rangle_g
\end{align*}
holds for all vector fields, $X, Y$ and $Z$.
\end{definition}

One can show (see lemma $5.2$ in \cite{lee2006riemannian}) that this condiiton is equivalent to
\begin{align*}
\frac{d}{dt} \langle V, W \rangle = \langle D_t V, W \rangle + \langle V, D_t W \rangle,
\end{align*}
for $V, W$ being vector fields along any curve $\gamma$.

The defintion of a symmetric connection involves the Lie bracket of two vector fields. If we think of vector fields on $M$, $X, Y$, as derivations acting on $C^\infty (M)$, then the Lie brakcet of $X$ and $Y$, $[X , Y]$, is the vector field (derivation) which acts on $f \in C^\infty (M)$ by
\begin{align*}
[X , Y] (f) := X(Y(f)) - Y(X(f)),
\end{align*}
where $X(f) \in C^\infty (M)$ is the function which evaluated at $p$ is the derivative of $f$ at $p$ in the direction of $X(p)$.

\begin{definition}
A connection on $M$ is \textit{symmetric} if
\begin{align*}
\nabla_X Y - \nabla_Y X \equiv [X, Y].
\end{align*}
\end{definition}

A symmetric connection is also called \textit{torsion free}, which corresponds to the vector fields along any curve not being "twisted" when they are parallel transported along the curve. By the Fundamental Theorem of Riemannian Geometry, given any Riemannian manifold $(M,g)$, there exists a unique connection, $\nabla$, on $M$ that is symmetric and compatible with $M$. This connection is called the Riemannian connection or the Levi-Civita connection, and geodesics with respect to this connection are called Riemannian geodesics. Since this choice of connection is unique, geodesics in a Riemannian manifold will always mean Riemannian geodesics. Such geodesics can be used to define the exponential map, $\exp: \; TM \rightarrow M$.

\begin{definition}
Given a point $p$ in a Riemannian manifold, $M$, and a vector $v \in T_p M$, the \textit{exponential map} is defined by $\exp_p (v) = \gamma_v(1)$ where $\gamma_v$ is the unique geodesic with $\gamma_v(0) = p$ and $\dot{\gamma}_v(0) = v$.
\end{definition}

The exponential map pushes a point $p \in M$ a unit distance in the direction of $v$ along the geodesic $\gamma_v$. Since $\dot{\gamma}_v(0) = v$, the differential of $\exp_p$ at $p$ is $v$, and so $\exp_p$ is a local diffeomorphism by the implicit function theorem. Given two points $p,q \in M$, the inverse of the exponential map, $\exp_p^{-1}(q) = \log_p(q) = \overrightarrow{pq}$, yields the element of $v \in T_p M$ for which $\gamma_v (1) = q$ \\[0.2 cm]
A Riemannian manifold is said to be \textit{geodesically complete} if every un-exentedable geodesic is defined on all of $\R$, and it follows from the Hopf-Rinow theorem that a manifold is geodesically complete if and only if there exists a $p \in M$ such that $\exp$ is defined on all of $T_p M$.\\[0.2 cm]
If the geodesic $\gamma_v$ is defined on all of $\R$, one can investigate for which values of $t$, the extended geodesic, $\gamma_v(t) = \exp_p (t v)$ is still a geodesic from $p$ to $\exp_p (t v)$. If $\gamma_v$ is a geodesic up to time $t_0$ and not after $t_0$, then $t_0$ is called a \textit{cut point}. The set of all cut points of all geodesics starting at $p$ is called the \textit{cut locus} and is denoted $C(p)$.

\subsection{Curvature}

When defining statistical properties such as mean and variance on Riemannian manifolds, measuring geometric propoerties of the manifold is needed. It is therefore of interest to determine on which manifolds these measurements are identical - that is to determine which manifolds are (locally) isomorphic. One way of determining this is to find a local invariant property of a manifold which is preserved by isometries (such that measurements of length are preserved). This property will be the \textit{curvature}.

Let $M = \R^n$ by equipped with the Euclidean metric and consider a vector field $Z$. Given two other vector fields, $X$ and $Y$, we can now differentiate $Z$ first along $X$ and then along $Y$ by using the Riemannian connection;
\begin{align*}
\nabla_Y \nabla_X Z,
\end{align*}
and in the opposite order by $\nabla_X \nabla_Y Z$. If $R^n = R^2$ and $X$ and $Y$ where just vector fields corresponding to local coordinates then, by commutativity of second order derivatives,
\begin{align*}
\nabla_Y \nabla_X Z - \nabla_X \nabla_Y Z = \nabla_Y \partial_1 Z^k \partial_k - \nabla_X \partial_2 Z^k \partial_k = \partial_2 \partial_1 Z^k \partial_k - \partial_1 \partial_2 Z^k \partial_k = 0.
\end{align*}
But if $X$ and $Y$ are arbitrary vector fields, this identity does not necessarily hold, since
\begin{align*}
\nabla_Y \nabla_X Z - \nabla_X \nabla_Y Z = XYZ^k \partial_k - YXZ^k \partial_k = (XYZ^k - YXZ^k)\partial_k,
\end{align*}
where $XYZ^k = X \left( Y \left( Z^k \right) \right)$ to ease notation. The action of the Lie bracket of $X$ and $Y$ is recognized in the parenthesis, and thus for $\R^n$ the following identity holds;
\begin{align}\label{flat}
\nabla_Y \nabla_X Z - \nabla_X \nabla_Y Z = \nabla_{[X,Y]} Z.
\end{align}
Since this identity depends on the Levi-Civita connection, it holds for all manifolds isometric to $\R^n$, and manifolds for which (\ref{flat}) holds will be called \textit{flat} manifolds. To define the curvature of a manifold is then to determine how "un-flat" the manifold is, by considering the \textit{curvature transformation}, $R: \; \mathcal{T}(M) \times \mathcal{T}(M) \times \mathcal{T}(M) \rightarrow \mathcal{T}(M)$, defined by
\begin{align*}
R(X,Y)Z =  \nabla_Y \nabla_X Z - \nabla_X \nabla_Y Z - \nabla_{[X,Y]} Z,
\end{align*}
which is identically zero for flat manifolds. The curvature transformation can then be used to determine the curvature of a vector field by defining the \textit{curvature tensor}.

\begin{definition}
The curvature on a Riemannian manifold, $(M, g)$ is
\begin{align*}
Rm(X,Y,Z,W) = \langle R(X,Y)Z, W \rangle_g.
\end{align*}
\end{definition}

With the definitions of geodesics, the exponential map and curvature we are ready to define concepts such as mean and variance of a random variable with values in a finite dimensional Riemannian manifold. We will see that the behaviour of the curvature is crucial when exploring existence and uniqueness of mean points.





\documentclass[a4,danish]{article}

\usepackage{amssymb}
\usepackage{amsmath}
\usepackage{amsthm}
\usepackage{xcolor}
\usepackage{soul}
\usepackage{enumerate}

\newtheoremstyle{break}
	{\topsep}{\topsep}
	{\bfseries}{}
	{\newline}{}
\theoremstyle{break}
\newtheorem{theorem}[subsection]{Theorem}
\newtheorem{lemma}[subsection]{Lemma}
\newtheorem{proposition}[subsection]{Proposition}
\newtheorem{corollary}[subsection]{Corollary}
\theoremstyle{definition}
\newtheorem{definition}[subsection]{Definition}
\newtheoremstyle{Break}
	{\topsep}{\topsep}
	{}{}
	{\bfseries}{}
	{\newline}{}
\theoremstyle{Break}
\newtheorem{example}[subsection]{Example}
\newtheorem{remark}[subsection]{Remark}
\newtheorem{note}[subsection]{Note}
\setcounter{secnumdepth}{0}
\usepackage{xpatch}
\xpatchcmd{\proof}{\ignorespaces}{\mbox{}\\\ignorespaces}{}{}


\newcommand{\Z}{\mathbb{Z}}
\newcommand{\Q}{\mathbb{Q}}
\newcommand{\R}{\mathbb{R}}
\newcommand{\N}{\mathbb{N}}
\newcommand{\C}{\mathbb{C}}
\renewcommand{\S}{\mathbb{S}}
\renewcommand{\P}{\text{P}}

\renewcommand{\phi}{\varphi}
\renewcommand{\epsilon}{\varepsilon}

\newcommand*\diff{\mathop{}\!\mathrm{d}}

\setlength{\parskip}{1em}
\setlength{\parindent}{0em}

% Figures -- use this instead of full file path because of git.
\usepackage{graphicx}
\graphicspath{{../figures/}}

\begin{document}
% \maketitle

\section*{Mean and variance}
\label{sec:mean_and_variance}

In order to perform statistics on shapes we must first try to define central statistical concepts on manifolds. In this section we focus on a geodsically complete Riemannian manifold, $(M, g)$, of dimension $n$, and present ways of defining the mean, variance and covariance of $M$-valued random variables. Given an underlying probability space, $(\Omega, \mathcal{F}, P)$, a $M$-valued random variable is a $\mathcal{F}/\mathcal{B}(M)$ measurable map, $X: \, \Omega \rightarrow M$, and we denote by $x = X(\omega)$ a realization of $X$ on $M$.\\[0.2 cm]
In order to perform statistics on $M$ we need to construct a measure on $M$. This measure is induced by the metric $g$ in the following way. Let $x = (x^1, \ldots , x^n)$ be representation of $x \in M$ in local coordinates, and let $\frac{\partial}{\partial x} = (\frac{\partial}{\partial x^1}, \ldots , \frac{\partial}{\partial x^n})$ be the corresponding basis of $T_x M$. The metric $g$ is then expressed in this basis by the matrix $G = [g_{ij}(x)]$ where $g_{ij}(x) = \langle \frac{\partial}{\partial x^i} , \frac{\partial}{\partial x^j} \rangle = g\left(\frac{\partial}{\partial x^i}, \frac{\partial}{\partial x^j}\right)$. The measure on $M$ is then defined by \hl{ $d M(x) = \sqrt{\left| \det G(x) \right|} dx$ }. $X$ is said to have density $p_X$ w.r.t. $d M$ if
\begin{align*}
P(X \in \mathcal{A}) = \int_{\mathcal{A}} p_X(y) d M(y),
\end{align*}
holds for all $\mathcal{A} \in \mathcal{B}(M)$ and if the integral over $M$ is equal to $1$. Here
$p_X$ is a density in the usual sense. It is a real-valued, positive and integrable function. If $\pi$ is a chart of the manifold, then $r := \pi(X(\omega))$ defines a random vector with density, $\rho_r$, w.r.t to the Lebesgue measure given by $\rho_r (y) = p_X (y) \sqrt{\left| \det G(y) \right|}$. If $\phi: \, M \rightarrow \R$ is a $\mathcal{B}(M)/ \mathcal{B}(\R)$-measurable map, then $\phi(X)$ deifnes a real-valued random variable for which the expection is
\begin{align*}
\mathbb{E} (\phi(X)) = \int_M \phi(y) p_X(y) d M(y).
\end{align*}
Unfortunately, we cannot define the expectation of $M$-valued random variables in a smiliar manner, since the real-valued integral does not generalize to an integral with values on $M$. Instead we generalize the notion of mean value by first defining the variance of a $M$-valued random variable and then defining the so-called \textit{Fréchet means} as minimizers of the variance. 




\end{document}


\documentclass[a4,danish]{article}

\usepackage{amssymb}
\usepackage{amsmath}
\usepackage{xcolor}
\usepackage{soul}
\usepackage{enumerate}

\newcommand{\Z}{\mathbb{Z}}
\newcommand{\Q}{\mathbb{Q}}
\newcommand{\R}{\mathbb{R}}
\newcommand{\N}{\mathbb{N}}
\newcommand{\C}{\mathbb{C}}
\renewcommand{\S}{\mathbb{S}}
\renewcommand{\P}{\text{P}}

\renewcommand{\phi}{\varphi}
\renewcommand{\epsilon}{\varepsilon}

\newcommand*\diff{\mathop{}\!\mathrm{d}}

\setlength{\parskip}{1em}
\setlength{\parindent}{0em}

\title{The tangent space of closed curves in $\R^2$}
\author{Mads and Anders}
\date{\today}

\begin{document}
% \maketitle

\section*{The tangent space of closed curves in $\R^2$}
\label{sec:tangent-space-closed}


\paragraph{Constructing the space of closed curves.}
Intuitively, we want to consider the space of all (smooth) closed
curves in $\R^2$. This can be seen as the space of all submanifolds in
$\R^2$ which are diffeomorphic to the unit circle $\S^1$. If we let
$\text{Imm}(\S^1,\R^2)$ denote the space of all \textit{immersion}
from the inut circle into the plane, we can define the space we
want to consider as
\begin{equation*}
  B :=
  \left\{
    q(\S^1) \mid q \in \text{Imm}(\S^1,\R^2)
  \right\}.
\end{equation*}
Here we simply think of $q(\S^1) \subset \R^2$ as a subspace and forget
about the actual map $q$. (Keeping this mapping in mind, we could
define the space in another way; but this is not so important right
now.)

\paragraph{The tangent space of B.}
For ordinary finite dimensional manifolds $B$, Lee defines the tangent
space at a point $p \in B$ through the notation of
\textit{derivations}; this is a rather abstract construction, but is
nice to work with. Using this, one can define the notion of a
tanget vector to a curve in $B$ passing through the point $p$. Then,
one can define an equivalence relation on the space of such curves and
obtain an equivalent definition of the tanget space, which is more
intuitive. One can also work the other way around

...

Imagine for a moment that
one way to define the tangent space $T_p$ at a point $p \in B$, is to
consider tangent vectors to curves/paths passing through the point
$p$. If $\gamma \colon [0,1] \rightarrow B$ is some path in $B$ we
consider
\begin{equation*}
  \gamma'(x)
  % := \frac{\partial }{\partial t} \gamma(t) \bigg\rvert_{t=x}
  := \frac{\partial }{\partial t} \bigg\rvert_{t=x} \gamma(t),
\end{equation*}
and define and equivalence relation as $\gamma_1 \sim \gamma_2$
whenever $\gamma'_1(0)=\gamma'_2(0)$

...

Taking for granted that $B$ actually is a (Fr\'echet) manifold, we now
want to determine what the tangent spaces look like.

\paragraph{Question / considerations / imprecisions.}






\end{document}
\subsection*{The L2 metric vanishes}
\label{sec:l2-metric-vanishes}

\hl{Some introductory text...}

Definition of the length of the path $\phi$ directly from the path alone:
\begin{equation*}
  L(\phi) = \int_{0}^{1} \int_{\S^{1}}
  \frac{\langle\phi_{t},i \phi_{\theta}\rangle^2}{|\phi_{\theta}|}  \diff \theta \diff t
\end{equation*}
Rewrite this as
\begin{equation*}
  \int_{0}^{1} \int_{\S^{1}}
  \left(
    \frac{\langle\phi_{t},i
      \phi_{\theta}\rangle}{|\phi_{t}||\phi_{\theta}|}
  \right)^2
  |\phi_t|^2   |\phi_{\theta}|
  \diff \theta \diff t
  =
  \int_{0}^{1} \int_{\S^{1}}
  \cos(\alpha(\phi_t, i\phi_{\theta}))^2
  |\phi_t|^2   |\phi_{\theta}|
  \diff \theta \diff t,
\end{equation*}
with $\alpha(x,y)$ denoting the angle between $x$ and $y$. When
constructing a zigzag-path the angle will be \hl{(approximately?)}
constant in $\theta$ and $t$,
and it is given by
\begin{equation*}
  n = \tan(\alpha).
\end{equation*}
We have that
\begin{equation*}
  \cos(\arctan(n)) = (1+n^2)^{-1/2}
  = O(n^{-1}),
\end{equation*}
so we can write
\begin{equation*}
  L(\phi) = O(n^{-2})
  \int_{0}^{1} \int_{\S^{1}}
  |\phi_t|^2   |\phi_{\theta}|
  \diff \theta \diff t,
\end{equation*}
in this case. To show that such a zigzag path has arbitrary
small length we just need to show that the integral does not grow
faster than $n^2$.

As an example, take the simply case where we expand the circle $e^{i\pi\theta}$ to
$2e^{i\pi\theta}$. The zigzag path in then concretely given as
\begin{equation*}
  \phi(t,\theta) = e^{i\pi\theta}
    \sum_{k=0}^{n-1}
    h^{n,k}(t,\theta) + g^{n,k}(t,\theta)
\end{equation*}
where
\begin{equation*}
  \begin{aligned}
    h^{n,k}(t,\theta) & := 1_{[\frac{k}{n},\frac{k}{n} +
      \frac{1}{2n})}(\theta) \left( 1+2t(n\theta-k) \right), \\
    g^{n,k}(t,\theta) & := 1_{[\frac{k}{n} + \frac{1}{2n},\frac{k+1}{n})}(\theta)
    \left( 1+2t(1-n\theta-k) \right).
  \end{aligned}
\end{equation*}

We have that
\begin{equation*}
  \begin{aligned}
    |\phi_t| & = \sum_{k=0}^{n-1} |h^{n,k}_t| + \sum_{k=0}^{n-1}
    |g^{n,k}_t|, \\
    |\phi_{\theta}| & = \sum_{k=0}^{n-1} |h^{n,k}_{\theta} + h^{n,k} | +
    \sum_{k=0}^{n-1}
    |g^{n,k}_{\theta} + g^{n,k} |,
  \end{aligned}
\end{equation*}
so by symmetry
\begin{equation*}
  \begin{aligned}
    \int_{0}^{1}
    |\phi_t|^2   |\phi_{\theta}|
    \diff \theta
    & =
    2n \int_{0}^{\frac{1}{2n}} |h^{n,0}_t|^2 |h^{n,0}_{\theta} + h^{n,0} |
    \diff \theta \\
    & = 2n \int_{0}^{\frac{1}{2n}}
    (2n\theta)^2(2tn+1+t2n\theta) \diff \theta \\
    & = \int_{0}^1
    u^2(2tn+1+t u) \diff \theta \\
    & = O(n),
  \end{aligned}
\end{equation*}
for $t\in[0,1]$ which gives the result.

%%% Local Variables:
%%% mode: latex
%%% TeX-master: "local-masters/local-main-L2-metric"
%%% End:

\subsection{Almost local metrics}
\label{sec:al-metrics}

The challenge at hand is the following: we wish to construct a notion of distance between two points in $\mathcal{I}$ by defining a metric, such that the distance between two points is the length of geodesics between the points. As we have seen, the distance induced by the $L^2$-metric vanishes on $\mathcal{I}$, so we seek to define metrics, which do not vanish. One type of such metrics is $\textit{almost local metrics}$, which, given $f \in$ Imm, are metrics of the form
\begin{align*}
G_f^\Phi (h,k) = \int_{S^1} \Phi(\text{Vol}(f), H_f, K_f) \bar{g}(h,k) \text{vol}(f^* \bar{g}),
\end{align*}
where $\Phi: \; \R^3 \rightarrow \R_{> 0}$ is smooth, $\text{Vol}(f) = \int_{S^1} \text{vol}(f^* \bar{g})$ is the total volume of $f(S^1)$, $H_f$ is the mean curvature of $f$ and $K_f$ is the Gauss curvature of $f$. Both $H_f$ and $K_f$ are local invariant properties with respect to the Riemanninan metric, defined to be the trace and the determinant of the Weingarten mapping, respectively, and so $\Phi$ is often chosen to only depend on one of the two curvatures. In the case of shapes in $\R^2$, $H_f(\theta) = \frac{\det(f_\theta, f_{\theta \theta})}{\left| f_\theta \right| ^3}$, which is just the usual formula for curvature of a plane curve.

$\Phi$ can also be seen as map from Imm$(S^1, \R^2)$ to $C^\infty (S^1, \R_{> 0})$. When viewed as such, in order for the metric to be invariant under reparametrizations, $\Phi$ must also be equivariant with respect to the action of the diffeomorphism group, Diff$(S^1)$ - i.e. $\Phi(f \circ \phi) = \Phi(f) \circ \phi$ for $\phi \in$ Diff$(S^1)$.

The total volume of $f$, $\text{Vol}(f)$, is defined via the volume form induced by the pullback metric, $f^*\bar{g}$, so this definition of almost-local metrics only applies to the manifold of embeddings from manifolds which admits a volume form. All compact, oriented manifolds do this, such as $S^1$, (Lemma $3.2$ in \cite{lee2006riemannian}), and almost local metrics are often defined for embeddings from this class of manifolds to $\R^n$. In the case of shapes in $R^2$, the volume form on $S^1$ induced by $f$ is given by vol$(f^*\bar{g}) = \left| f_\theta \right| d \theta$ (section 2.2 in \cite{michor2003riemannian}). In our case, almost local metrics therefore take on the form
\begin{align*}
G_f^\Phi (h,k) = \int_{S^1} \Phi(\text{Vol}(f), H_f, K_f) \bar{g}(h,k) \left| f_\theta \right| d \theta.
\end{align*}
Vol$(f)$ is a non-local property of $f$, and thus the metrics are not only dependent on the local properties, $K_f, H_f$, but must be $\textit{almost}$ local metrics.

\begin{remark}
Both curvatures and the volume form of a shape in $\R^2$ take on a particular nice form, but expressions can also be found for the general case where $f \in$Imm$(M, \R^n)$ with $M$ a compact orientable $n-1$ dimensional manifold. This is done by using the Levi-Civita connections of the Riemannian manifolds $(\R^n,\bar{g})$ and $(M,G^\Phi)$ to construct the Weingarten mapping (see sections $3.4$ and $3.9$ of \cite{bauer2010almost}).
\end{remark}
Note that if $\Phi$ depends only on $f$ through Vol$(f)$ then $G_f^\Phi (h,k)$ is equal to the L$^2$-metric (up to a constant). But if $\Phi$ actually depends on either curvature and the total volume, then point-separation is achieved under certain conditions imposed on $\Phi$;

\begin{theorem}\label{point_sep}
If $\Phi(\text{Vol} \, (f), H_f, K_f) = 1 + A H_f^2$ for some $A > 0$, then $G_f^\Phi$ induces a point-separating metric on $\mathcal{I}$.
\end{theorem}

\begin{proof}
We here sketch the ideas of the proof as found in section 3 of \cite{michor2003riemannian}.

Given a path of un-parametrized shapes, $\pi(q) \, : \, [0,1] \times S^1 \rightarrow \mathcal{I}$, one can choose a path, $q$, in Imm$(S^1, \R^2)$ such that $q(0, \cdot)$ is an immersion of constant speed, $\langle q_t, q_\theta \rangle = 0$ for all $t$ and $\theta$, and $q(t , \theta)$ has constant speed. Let $q$ be such a path, and let
\begin{align*}
\Phi(f) = 1 + A H_f ^2
\end{align*}
for some constant $A > 0$. Consider the Hilbert space $L^2(S^1, \left| q_\theta(t, \theta) \right| d\theta) = L^2(S^1, \text{vol} (q(t)^* \bar{g}))$. The Cauchy-Schwarz inequality yields
\begin{align*}
\int_{S^1} \left| q_t(t, \theta) \right| \left| q_\theta(t, \theta) \right| d \theta \leq \left(\int_{S^1} \left| q_\theta(t, \theta) \right| d \theta \right) ^{\frac{1}{2}} \left( \int_{S^1} \left| q_t(t, \theta) \right|^2 \left| q_\theta(t, \theta) \right| d \theta   \right) ^{\frac{1}{2}}.
\end{align*}
The length of the path $q$ is then
\begin{align*}
L_{G^{\Phi}}(q) := \int_0^1  \sqrt{G^{\Phi}_{q(t)}(q_t, q_t)} dt &= \int_0^1 \left( \int (1 + AH_{q(t)}^2 ) \left| q_t(t, \theta) \right|^2 \left| q_\theta(t, \theta) \right| d\theta     \right) ^{\frac{1}{2}} dt \\
& \geq \int_0^1 \left(\int_{S^1} \left| q_\theta(t, \theta) \right| d \theta \right) ^{-\frac{1}{2}} \int_{S^1} \left| q_t(t, \theta) \right| \left| q_\theta(t, \theta) \right| d \theta dt.
\end{align*}
The mean value theorem for integrals then yields that there exists $t_0 \in [0, 1]$ such that
\begin{align*}
L_{G^{\Phi}}(q)\geq  \left(\int_{S^1} \left| q_\theta(t_0, \theta) \right| d \theta \right) ^{-\frac{1}{2}} \int_0^1 \int_{S^1} \left| q_t(t, \theta) \right| \left| q_\theta(t, \theta) \right| d \theta dt,
\end{align*}
where the first factor is the curve length of $q(t_0, \cdot)$ to the power of $-\frac{1}{2}$, and the second factor is the area in $\R^2$ swept out by the path $q$ (see Figure \ref{fig:area-swep}). We note that if the shape is not trivially a point in $\R^2$ (such that the length at time $t_0$ is $0$) and if the path is not trivial (such that $q(0, \cdot) = q(1, \cdot)$), then this lower bound is strictly positive. Thus any path from two distinct shapes have length greater than $0$, such that the metric induces a point-separating distance function.
\end{proof}

\begin{figure}[h!]
  \centering
    \includegraphics[scale = 1]{deform_circle.pdf}
  \caption{Illustration of the area swept out by a path $q$ in $\mathcal{I}$ with $q(0) = S^1$.}
  \label{fig:area-swep}
\end{figure}

No matter the choice of $\Phi$, an almost local metric is never point-separating on Imm$(S^1, \R^2)$ - the shape space without quotienting out reparametrizations. To see this let $f \in$ Imm$(S^1, \R^2)$ and take $\tilde{f}$ to be in the orbit of $f$ of the Diff$(S^1)$ action - i.e. $\tilde{f} = f \circ \phi$ for some $\phi \in$ Diff$(S^1)$. Since $\Phi$ is equivariant w.r.t. the action of Diff$(S^1)$,
\begin{align*}
G_{\tilde{f}}^\Phi (h,k) = \int_{S^1} \Phi(\tilde{f}) \bar{g}(h,k) \text{vol}(f^* \bar{g}) = \int_{S^1} \Phi(f) \circ \phi \bar{g}(h,k) \text{vol}(f^* \bar{g}),
\end{align*}
so if one chooses suitable diffeomorphisms $\phi$ then $\Phi(f)$ is a constant element of $C^\infty(S^1, \R_{> 0})$. The length of a path from $f$ to $\tilde{f}$ in the orbit of $f$ is then bounded by
\begin{align*}
\mathrm{L}_{G^\Phi}(q) & = \int_0^1 \int_{S^1} \Phi(f) \circ \phi \bar{g}(h,k) \text{vol}(f^* \bar{g}) dt \\ 
& = \int_0^1 \left( \int_{S^1} \Phi(f) \circ \phi \vert q_t(t, \theta) \vert ^2 \vert q_\theta(t, \theta) \vert d \theta d \theta \right) ^{\frac{1}{2}} dt \\
& \leq \max_{\theta \in S^1} \Phi(f) \int_0^1 \left( \int_{S^1} \vert q_t(t, \theta) \vert ^2 \vert q_\theta(t, \theta) \vert d \theta d \theta \right) ^{\frac{1}{2}} dt, 
\end{align*}
which is a weighted version of the $L^2$ metric. As the geodesic distance function induced by weighted $L^2$ metrics vanishes, the almost local metric vanishes for points in Imm$(S^1, \R^2)$ which are in the same orbit of the Diff$(S^1)$-action.

In general, existence and uniqueness of geodesics w.r.t. almost local metrics are not ensured and thus the length of a path in $\mathcal{I}$ cannot be determined by constructing a geodesic and computing its length (section 5.2 in \cite{bauer2014overview}). In certain cases however, the length of a path is exactly the lower bound derived in \ref{point_sep} (see theorem 3.1 in \cite{shah2007h0type}).

\begin{example}
Define an almost local metric on $\mathcal{I}$ as above with $\Phi(f) = \ell(f)$ where $\ell(f)$ is the ordinary curve length of $f$ (which implicit is a function of the curvatures of $f$). Let $f, g \in \mathcal{I}$ be shapes and let $q \, : \, [0,1] \rightarrow \mathcal{I}$ be a path from $f$ to $g$ such that $q(0) = f$ and $q(1) = g$. The length of the path $q$ is then the area swept out by $q$ in $\R^2$,
\begin{align*}
L_{G^\Phi}(q) = \int_{[0,1]} \int_{S^1} \left| q_t(t, \theta) \right| \left| q_\theta(t, \theta) \right| d\theta dt,
\end{align*}
and the distance between $f$ and $g$ is then the infimum over all paths in $\mathcal{I}$ which start in $f$ and end in $g$:
\begin{align*}
d_{G^\Phi}(f, g) = \inf_{q\in\mathcal{Q}} \int_{[0,1]} \int_{S^1} \left| q_t(t, \theta) \right| \left| q_\theta(t, \theta) \right| d\theta dt,
\end{align*}
where $\mathcal{Q}$ denotes all paths $q$, such that $q(0) = f$ and $q(1) = g$.
\end{example}
\begin{example}
If $\Phi$ is a more general function of the curve length, $\Phi = e^{A \ell (f)}$, for some constant $A > 0$, then the distance between two shapes, $f$ and $g$, is bounded by
\begin{align*}
\inf_{q\in\mathcal{Q}} \sqrt{A e} \int_{[0,1]} \int_{S^1} \left| q_t(t, \theta) \right| \left| q_\theta(t, \theta) \right| d\theta dt \leq d_{G^\Phi}(f, g) \\
 \leq \inf_{q\in\mathcal{Q}} \sqrt{A e} K \int_{[0,1]} \int_{S^1} \left| q_t(t, \theta) \right| \left| q_\theta(t, \theta) \right| d\theta dt,
\end{align*}
with $K = e^{A \ell_{max}(q) / 2}$ where $\ell_{max} (q) = \max_{t \in [0,1]} \ell (q(t, \cdot))$ is the maximum length of any immersion on the path from $f$ to $g$. In particular, if $f \neq g$, such that there exists no trivial path between the two shapes, then the distance is positive, since the area swept out in $\R^2$ by any non-trivial path is positive.
\end{example}
Almost local metrics seem to be solutions to the challenge of finding a non-vanishing distance function on $\mathcal{I}$. After equipping $\mathcal{I}$ with such a distance function it is natural to wonder if we can in some way define the statistical concepts of section \ref{sec:mean_and_variance} on $\mathcal{I}$. We finish this exposition by a discussion on the challenges of trying to define these concepts on this infinite dimensional manifold.





\section{Statistical concepts on shape space}
\label{sec:statistical_concepts_on_shape_space}

If we wish to define statistical concepts like mean and variance on our shape space - the manifold of unparametrized curves - we are faced with some challenges. To begin with our manifold fold is infinite-dimensional, and all theory developed so far in this exposition has been on finite-dimensional manifolds. This is an important part of the generalizations of mean and variance, since we can quite easily define a measure on $M$. If a point $x \in M$ can be written in finitely many local coordinates, a measure can be defined via the use of the metric on the induced (finite) basis of $T_x M$. It is not at all obvious how this method of constructing a measure on $M$ generalizes to the infinite-dimensional case.

For example, the prime example of a infinite-dimensional shape space is a function space, e.g., the space of all continuous function. As is well-known, simply constructing a random variable on such a space is non-trivial; a fundamental construction is the Brownian motion, the existence of which relies on a deep result of Kolmogorov. Our space $\I$ is an example of a restriction of a space of continuous functions (to immersions), and the space $\mathcal{I}$ is even more involved. Thus, simply defining the random points to work with on the shape space promises some challenges.

Secondly, all results regarding existence and uniqueness of mean points have relied on either a global assumption or local assumptions on the Riemannian curvature of the manifold. The behaviour of the curvature of our shape space is not at all well understood, especially since the choice of metric is non-trivial to begin with. One might think that the curvature of $\mathcal{I}$ equipped with the $L^2$-metric is the least difficult to examine, but since the distance function induced by this metric is vanishing, the resulting variance, $\sigma^2_X(y)$ would be $0$ for all random variables $X$ and shapes $y$ (if it is even possible to generalize the variance-formulas in section \ref{sec:mean_and_variance} to the infinite dimensional case.). Thus every point would be a mean point of $X$ which is not a very informative statement.

To make the difficulties stand out explicitly, we write our Definition~\ref{sec:mean_and_variance} when naively replacing the finite-dimensional manifold $M$ with our shape space $\I$ and using an almost local metric. Then we have that $y$ is some fixed shape and $X$ is a random shape (still not defined):
\begin{align*}
  \sigma^2_X(y) &  = \int_{\I}
                  \inf_{c \in \mathcal{C}_{y}^z}
                  \left\{
                  L_{G^\Phi}(c)
                  \right\}
                  p_X(z)
                  \diff z
  % \\
  %               &  = \int_{\I}
  %                 \inf_{\stackrel{c \in \mathcal{C}}{c(0,\blank)=y, c(1,\blank)=z}}
  %                 \int_{\S^1} \Phi(\mathrm{Vol}(c(t,\blank),H_{c(t,\blank)},K_{c(t,\blank)}),
  %                 \overline{g}(c_t,c_t) |c_{\theta}| \diff \theta
  %                 \diff z \\
  ,
\end{align*}
where we let $\mathcal{C}_y^z$ denote all path starting in $y$ and ending in $z$.
First of all, it is not clear that this integral over a function space is well-defined.
Secondly, to calculate an explicit expression for the variance at $y$ (and later on the mean), we would most likely need to be able to find a useful expression for the distance $L_{G^{\Phi}}$, which is not trivial.
Thirdly, if we succeed in making sense of the above expression, the next step would be to consider how to minimize it, i.e., what we are actually interested in is
\begin{equation*}
  \argmin_{y \in \I}
  \int_{\I}
  \inf_{c \in \mathcal{C}_{y}^z}
  \left\{
    L_{G^\Phi}(c)
  \right\}
  p_X(z)
  \diff z.
\end{equation*}
This is optimization problem over a function space, so we could maybe use some tools from the calculus of variations to solve it.

All in all, at each step a lot of rather non-trivial problems need to be clarified, we imagine that one could proceed along the way sketched above.


%%% Local Variables:
%%% mode: latex
%%% TeX-master: "mainfile"
%%% reftex-default-bibliography: ("litteratur.bib")
%%% End:

\newpage
\section*{Appendix}
\label{sec:appendix}

An overview of which parts where contributed by whom. Section $3$ and subsection $4.1$ was contributed by Anders and sections $1, 2$ and subsection $4.2$ where contributed by Mads. The discussion in section $5$ is joint work with Mads contributing every even line and Anders contributing every odd line.

\newpage

\bibliography{litteratur.bib}{}

\end{document}
