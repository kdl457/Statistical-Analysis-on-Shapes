\documentclass[a4,danish]{article}

\usepackage{amssymb}
\usepackage{amsmath}
\usepackage{amsthm}
\usepackage{xcolor}
\usepackage{soul}
\usepackage{enumerate}

\newtheoremstyle{break}
	{\topsep}{\topsep}
	{\bfseries}{}
	{\newline}{}
\theoremstyle{break}
\newtheorem{theorem}[subsection]{Theorem}
\newtheorem{lemma}[subsection]{Lemma}
\newtheorem{proposition}[subsection]{Proposition}
\newtheorem{corollary}[subsection]{Corollary}
\theoremstyle{definition}
\newtheorem{definition}[subsection]{Definition}
\newtheoremstyle{Break}
	{\topsep}{\topsep}
	{}{}
	{\bfseries}{}
	{\newline}{}
\theoremstyle{Break}
\newtheorem{example}[subsection]{Example}
\newtheorem{remark}[subsection]{Remark}
\newtheorem{note}[subsection]{Note}
\setcounter{secnumdepth}{0}
\usepackage{xpatch}
\xpatchcmd{\proof}{\ignorespaces}{\mbox{}\\\ignorespaces}{}{}


\newcommand{\Z}{\mathbb{Z}}
\newcommand{\Q}{\mathbb{Q}}
\newcommand{\R}{\mathbb{R}}
\newcommand{\N}{\mathbb{N}}
\newcommand{\C}{\mathbb{C}}
\renewcommand{\S}{\mathbb{S}}
\renewcommand{\P}{\text{P}}

\renewcommand{\phi}{\varphi}
\renewcommand{\epsilon}{\varepsilon}

\newcommand*\diff{\mathop{}\!\mathrm{d}}

\setlength{\parskip}{1em}
\setlength{\parindent}{0em}

% Figures -- use this instead of full file path because of git.
\usepackage{graphicx}
\graphicspath{{../figures/}}

\begin{document}
% \maketitle

\section*{Almost local metrics}
\label{sec:al-metrics}

The problem at hand is the following: we wish to construct a notion of distance between two points in $B_e(S^1,\R^2)$ by defining a metric, such that the distance between two points is the length of geodesics between the points. As we have seen, the distance induced by the $L^2$-metric vanishes on $B_e(S^1, \R^2)$, so we seek to define metrics, which do not vanish. One type of such metrics is $\textit{almost local metrics}$, which, given $f \in B_e(S^1, \R^2)$, are metrics of the form
\begin{align*}
G_f^\Phi (h,k) = \int_{S^1} \Phi(\text{Vol}(f), H_f, K_f) \bar{g}(h,k) \text{vol}(f^* \bar{g}),
\end{align*}
where $\Phi: \; \R^3 \rightarrow \R_{> 0}$ is smooth, $\text{Vol}(f) = \int_{S^1} \text{vol}(f^* \bar{g})$ is the total volume of $f(S^1)$, $H_f$ is the mean curvature of $f$ and $K_f$ is the Gauss curvature of $f$. Both $H_f$ and $K_f$ are local invariant properties with respect to the Riemanninan metric, defined to be the trce and the determinant of the Weingarten mapping, respectively, and so $\Phi$ is often chosen to only depend on one of the two curvatures. $\Phi$ can also be seen as map from Imm$(S^1, \R^2)$ to $C^\infty (S^1, \R_{> 0}$. When viewed as such, in order for the metric to be invariant under reparametrizations, $\Phi$ must also be equivariant with respect to the action of the diffeomorphism group, Diff$(S^1)$ - i.e. $\Phi(f \circ \phi) = \Phi(f) \circ \phi$ for $\phi \in$ Diff$(S^1)$. \\[0.2 cm]
The total volume of $f$, $\text{Vol}(f)$, is defined via the volume form induced by the pullback metric, $f^*\bar{g}$, so this definition of almost-local metrics only applies to manifolds which posses a volume form. All compact, oriented manifolds do this (\hl{Reference?}), and almost local metrics are often defined for this class of manifolds. In the case of $f \in B_e(S^1, \R^2)$, the volume form on $S^1$ induced by $f$, is given by vol$(f^*\bar{g}) = \left| f_\theta \right| d \theta$. (\hl{Reference to Riemannian Geometries on Spaces of Plane Curves 2.2}) \\[0.2 cm] 
Vol$(f)$ is a non-local property of $f$, and thus the metrics are not only dependent on the local properties, $K_f, H_f$, but must be $\textit{almost}$ local metrics. \\[0.2 cm]
Note that if $\Phi$ depends only on $f$ through Vol$(f)$ then $G_f^\Phi (h,k)$ is equal to the L$^2$-metric (up to a constant) (\hl{is this obvious from our definition of the L$^2$ metric ?}). But if $\Phi$ actually depends on either curvature and the total volume, then point-separation is achieved under certain conditions imposed on $\Phi$;

\begin{theorem}\label{point_sep}
If $\Phi(\text{Vol} \, (f), H_f, K_f) \geq A H_f$ for some $A > 0$, then $G_f^\Phi$ induces a point-separating metric on $B_e(S^1, \R^2)$.
\end{theorem}

\begin{proof}
\hl{Reference - perhaps explain heuristically?}
\end{proof}

No matter the choice of $\Phi$, an almost local metric is never point-separating on Imm$(S^1, \R^2)$ - the shape space without quotienting out reparametrizations. To see this let $f \in$ Imm$(S^1, \R^2)$ and take $\tilde{f}$ to be in the orbit of $f$ of the Diff$(S^1)$ action - i.e. $\tilde{f} = \phi \circ f$ for some $\phi \in$ Diff$(S^1)$. Since $\Phi$ is equivariant w.r.t. the action of Diff$(S^1)$, 
\begin{align*}
G_{\tilde{f}}^\Phi (h,k) = \int_{S^1} \Phi(\tilde{f}) \bar{g}(h,k) \text{vol}(f^* \bar{g}) = \int_{S^1} \Phi(f) \circ \phi \bar{g}(h,k) \text{vol}(f^* \bar{g}),
\end{align*}
the almost local metric restricted to the orbit of $f$ can be viewed as a weighted $L^2$-type metric with weights represented by $\Phi(f) \circ \phi$. As the geodesic distance function induced by weighted $L^2$ metrics vanishes (\hl{Reference or follows easily from proof?}), the almost local metric vanishes for point in Imm$(S^1, \R^2)$ which are in the same orbit of the Diff$(S^1)$-action.
\\[0.2 cm]
In general, existence and uniqueness of geodesics w.r.t. almost local metrics are not ensured and thus the length of a path in $B_e(S^1,\R^2)$ cannot be determined by constructing a geodesic and computing its length. In certain cases however, the length of a path is exactly the lower bound used in \ref{point_sep} \hl{Reference to theorem 3.1 in H0 type Riemannian metrics on the space of planar curves}.

\begin{example}
Define an almost local metric on $B_e(S^1, \R^2)$ as above with $\Phi(f) = \ell(f)$ where $\ell(f)$ is the ordinary curve length of $f$ (which implicit is a function of the curvatures of $f$). Let $q_0, q_1 \in B_e(S^1, \R^2)$ be shapes and let $c \, : \, [0,1] \rightarrow B_e(S^1, \R^2)$ be a path from $q_0$ to $q_1$ such that $c(0) = q_0$ and $c(1) = q_1$. The length of the path $c$ is then the area swept out by $c$ in $\R^2$,
\begin{align*}
L_{G^\Phi}(c) = \int_{S^1} \int_{[0,1]} \left| \det dc(t, \theta) \right| dt d\theta,
\end{align*}
and the distance between $q_0$ and $q_1$ is then the infimum over all paths in $B_e(S^1, \R^2)$ which start in $q_0$ and end in $q_1$:
\begin{align*}
d_{G^\Phi}(q_0, q_1) = \inf_{c\in\mathcal{C}} \int_{S^1} \int_{[0,1]} \left| \det dc(t, \theta) \right| dt d\theta,
\end{align*}
where $\mathcal{C}$ denotes all paths $c$, such that $c(0) =q_0$ and $c(1) =q_1$. \\[0.2 cm]
\end{example}

\begin{example}
If $\Phi$ is a more general function of the curve length, $\Phi = e^{A \ell (f)}$, for some constant $A > 0$, then the distance between two shapes, $q_0$ and $q_1$, is bounded by
\begin{align*}
\inf_{c\in\mathcal{C}} \sqrt{A e} \int_{S^1} \int_{[0,1]} \left| \det dc(t, \theta) \right| dt d\theta \leq d_{G^\Phi}(q_0, q_1) \leq \inf_{c\in\mathcal{C}} \sqrt{A e} e^{A \ell_{max}(c) / 2} \int_{S^1} \int_{[0,1]} \left| \det dc(t, \theta) \right| dt d\theta,
\end{align*}
where $\ell_{max} (c) = \max_{t \in [0,1]} \ell (c(t, \cdot))$ is the maximum length of any immersion on the path from $q_0$ to $q_1$. In particular, if $q_0 \neq q_1$, such that there exists no trivial path between the two shapes, then the distance is positive, since the area swept out in $\R^2$ by any path is positive. 
\end{example}



\end{document}



