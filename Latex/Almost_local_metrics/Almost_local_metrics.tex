\documentclass[a4,danish]{article}

\usepackage{amssymb}
\usepackage{amsmath}
\usepackage{amsthm}
\usepackage{xcolor}
\usepackage{soul}
\usepackage{enumerate}

\newtheoremstyle{break}
	{\topsep}{\topsep}
	{\bfseries}{}
	{\newline}{}
\theoremstyle{break}
\newtheorem{theorem}[subsection]{Theorem}
\newtheorem{lemma}[subsection]{Lemma}
\newtheorem{proposition}[subsection]{Proposition}
\newtheorem{corollary}[subsection]{Corollary}
\theoremstyle{definition}
\newtheorem{definition}[subsection]{Definition}
\newtheoremstyle{Break}
	{\topsep}{\topsep}
	{}{}
	{\bfseries}{}
	{\newline}{}
\theoremstyle{Break}
\newtheorem{example}[subsection]{Example}
\newtheorem{remark}[subsection]{Remark}
\newtheorem{note}[subsection]{Note}
\setcounter{secnumdepth}{0}
\usepackage{xpatch}
\xpatchcmd{\proof}{\ignorespaces}{\mbox{}\\\ignorespaces}{}{}


\newcommand{\Z}{\mathbb{Z}}
\newcommand{\Q}{\mathbb{Q}}
\newcommand{\R}{\mathbb{R}}
\newcommand{\N}{\mathbb{N}}
\newcommand{\C}{\mathbb{C}}
\renewcommand{\S}{\mathbb{S}}
\renewcommand{\P}{\text{P}}

\renewcommand{\phi}{\varphi}
\renewcommand{\epsilon}{\varepsilon}

\newcommand*\diff{\mathop{}\!\mathrm{d}}

\setlength{\parskip}{1em}
\setlength{\parindent}{0em}

% Figures -- use this instead of full file path because of git.
\usepackage{graphicx}
\graphicspath{{../figures/}}

\begin{document}
% \maketitle

\section*{Almost local metrics}
\label{sec:al-metrics}

The problem at hand is the following: we wish to construct a notion of distance between two points in $B_e(S^1,\R^2)$ by defining a metric, such that the distance between two points is the length of geodesics between the points. As we have seen, the distance induced by the $L^2$-metric vanishes on $B_e(S^1, \R^2)$, so we seek to define metrics, which do not vanish. One type of such metrics is $\textit{almost local metrics}$, which, given $f \in B_e(S^1, \R^2)$, are metrics of the form
\begin{align*}
G_f^\Phi (h,k) = \int_M \Phi(\text{Vol}(f), H_f, K_f) \bar{g}(h,k) \text{vol}(f^* \bar{g}),
\end{align*}
where $\Phi: \; \R^3 \rightarrow \R_{> 0}$ is smooth, $\text{Vol}(f) = \int_M \text{vol}(f^* \bar{g})$ is the total volume of $f(M)$, $H_f$ is the mean curvature of $f$ and $K_f$ is the Gauss curvature of $f$. Both $H_f$ and $K_f$ are local invariant properties with respect to the Riemanninan metric, defined to be the trce and the determinant of the Weingarten mapping, respectively, and so $\Phi$ is often chosen to only depend on one of the two curvatures. \\[0.2 cm]
The total volume of $f$, $\text{Vol}(f)$, is defined via the volume form induced by the pullback metric, $f^*\bar{g}$, so this definition of almost-local metrics only applies to manifolds which posses a volume form. All compact, oriented manifolds do this (\hl{Reference?}), and almost local metrics are often defined for this class of manifolds. In the case of $f \in B_e(S^1, \R^2)$, the volume form on $S^1$ induced by $f$, is given by vol$(f^*\bar{g}) = \left| f_\theta \right| d \theta$. (\hl{Reference to Riemannian Geometries on Spaces of PLane Curves 2.2}) \\[0.2 cm] 
Vol$(f)$ is a non-local property of $f$, and thus the metrics are not only dependent on the local properties, $K_f, H_f$, but must be $\textit{almost}$ local metrics. \\[0.2 cm]
Note that if $\Phi$ depends only on $f$ through Vol$(f)$ then $G_f^\Phi (h,k)$ is equal to the L$^2$-metric (up to a constant) (\hl{is this obvious from our definition of the L$^2$ metric ?}). But if $\Phi$ actually depends on either curvature and the total volume, then point-separation is achieved under certain conditions imposed on $\Phi$;

\begin{theorem}
If $\Phi(\text{Vol} \, (f), H_f, K_f) \geq A H_f$ for some $A > 0$, then $G_f^\Phi$ induces a point-separating metric on $B_e(S^1, \R^2)$.
\end{theorem}

\begin{proof}
\hl{Reference - perhaps explain heuristically?}
\end{proof}





\end{document}



