\documentclass[a4,danish]{article}

\usepackage{amssymb}
\usepackage{amsmath}
\usepackage{xcolor}
\usepackage{soul}
\usepackage{enumerate}
\usepackage{verbatim}

% Def, theorems, etc.
\usepackage{amsthm}
\newtheoremstyle{break}
	{\topsep}{\topsep}
	{\bfseries}{}
	{\newline}{}
\theoremstyle{break}
\newtheorem{theorem}{Theorem}[subsection]
\newtheorem{lemma}[subsection]{Lemma}
\newtheorem{proposition}[subsection]{Proposition}
\newtheorem{corollary}[subsection]{Corollary}
\theoremstyle{definition}
\newtheorem{definition}[subsection]{Definition}
\newtheoremstyle{Break}
	{\topsep}{\topsep}
	{}{}
	{\bfseries}{}
	{\newline}{}
\theoremstyle{Break}
\newtheorem{example}[subsection]{Example}
\newtheorem{remark}[subsection]{Remark}
\newtheorem{note}[subsection]{Note}
% \setcounter{secnumdepth}{0}
% \usepackage{xpatch}
% \xpatchcmd{\proof}{\ignorespaces}{\mbox{}\\\ignorespaces}{}{}


\newcommand{\Z}{\mathbb{Z}}
\newcommand{\Q}{\mathbb{Q}}
\newcommand{\R}{\mathbb{R}}
\newcommand{\N}{\mathbb{N}}
\newcommand{\C}{\mathbb{C}}
\newcommand*\I{\mathop{}\!\mathrm{Imm}}
\renewcommand{\S}{\mathbb{S}}
\renewcommand{\P}{\text{P}}
\newcommand{\blank}{\makebox[1ex]{\textbf{$\cdot$}}}
\renewcommand{\phi}{\varphi}
\renewcommand{\epsilon}{\varepsilon}

\newcommand*\diff{\mathop{}\!\mathrm{d}}

\setlength{\parskip}{1em}
\setlength{\parindent}{0em}

% Figures -- use this instead of full file path because of git.
\usepackage{graphicx}
\usepackage{caption}
\usepackage{subcaption}
\graphicspath{{../figures/}}

\title{The tangent space of closed curves in $\R^2$}
\author{Mads and Anders}
\date{\today}

\begin{document}
% \maketitle


\section{The manifold of curves}
\label{sec:manifold-curves}

A 2-dimensional shape can be thought of as a closed (smooth) curve in $\R^2$. Thus, we want to define a manifold structure on the space of these curves. Doing this mathematically correct is rather technical because we need to consider quotient spaces of infinite dimensional manifolds. In our exposition we shall to a large extend ``define our way out of this'' by, for example, defining tangent vectors of curves instead of deducing how these look like from the formal definition of the underlying manifold. We shall motivate our definitons geometrically and then deduce some properties from these definitons -- properties, which can also be deduced from the formal definitions. At the end of this section we briefly address what we miss with our more informal treatment.

A shape in $\R^2$ can either be thought of as a parametrized object, e.g., as a function $\S^1 \ni \theta \mapsto c(\theta) \in \R^2$; but it can also be thought of as an unparametrized object, e.g., the \textit{image} of such a function $\text{Im}(c) \subset \R^2$ (as illustrated in Figure~\ref{fig:circle-mapping}). We impose some smoothness structure on the curves and define the spaces we want to consider formally as
\begin{equation*}
  \begin{aligned}
    \I & := \I(\S^1, \R^2)  :=  \{ ...? \}, \\
    \mathcal{I} & := \mathcal{I}(\S^1, \R^2)  := \I(\S^1, \R^2)/\mathrm{Diff}(\S^1).
  \end{aligned}
\end{equation*}
$\I$ is the space of \textit{immersion} of the unit circle into $\R^2$, and $\mathcal{I}$ is then this space modulo reparametrization, i.e., we identify two object $q,p \in \I$ in $\mathcal{I}$ if $q=p \circ \phi$, where $\phi \in \mathrm{Diff}(\S^1)$ is a diffeomorphism on the unit circle.

When our interest is upon the \textit{shape}, we are not really interested in the underlying parametrization of this shape, and so we are mostly interested in the space $\mathcal{I}$. However, it is easiest to construct a manifold structure on $\I$ and then deduce one on $\mathcal{I}$, so we start by considering the parametrized space of immersions. 

\begin{figure}
  \label{fig:circle-mapping}
  \centerline{\includegraphics[width=0.9\linewidth]{circle_mapping.pdf}}
  \caption{A simple shape. We can think of a shape as the whole mapping, or simply as the subset to the right.}
\end{figure}

\subsection*{The manifold of parametrized curves -- $\I(\S^1, \R^2)$}
\label{sec:parametrized-curves}

The first structure we want to impose on our manifold of curves is tangent spaces and tangent vector to our elements in the space.

For ordinary finite dimensional manifolds $M$ \hl{we have so far worked with} the definition of tangents vectors as \textit{derivatives}. This is a rather abstract construction, which, however, turns out to be nice to work with. Fortunalety, we know that this definition corresponds to the more geometrically intuitive definition of tangent vectors at a point $m \in M$ as derivatives of paths going through $m$. We shall use this a motivation for our definition of tangent vectors to points in $\I$.

Consider a point $c\in \I$ and a path in $\I$ defined around $0$ that goes through the point $c$. This path is a map
\begin{equation}
  \label{eq:path-in-imm}
  [-\epsilon, \epsilon] \ni t \mapsto  q(t, \blank) :=  (\theta\mapsto q(t, \theta)) \in \I,
  \quad q(0,\theta) = c(\theta),
\end{equation}
i.e., for each $t$ we get a parametrized curve, and at 0 we get the curve $c$. We can also think of this a (smooth) map
\begin{equation*}
  [-\epsilon, \epsilon] \times \S^1  \ni (t, \theta) \mapsto q(t, \theta) \in \R^2.
\end{equation*}
As in the finite dimensional case we can now take the \textit{time derivate} of our path and evaluate this at 0; of course, this is technically not obvious, as our constructed path maps into a function space, but we shall here simply take this derivative to be understood pointwise:

\begin{definition}
  The tangent space $T_{c}\I$ to an element $c \in \I$ consists of all function
  $h\colon \S^1 \rightarrow \R^2$ such that
  \begin{equation*}
    h(\theta) = q_t(0, \theta) := \frac{\partial }{\partial t} \bigg\vert_{t=0} q(t,\theta),
  \end{equation*}
  where $q$ is some path passing through $c$ at 0 (as defined in \eqref{eq:path-in-imm}).
\end{definition}

By this definition, we can essentially think of tangent vectors in $\I$ as vector fields on the circle. Figure~\ref{fig:def-tang-imm} illustrates the idea behind the definition. 

\begin{figure}
  \centering
  \begin{subfigure}{.49\textwidth}
    \centering
    \includegraphics[width=1\linewidth]{path.pdf}
  \end{subfigure}
  \begin{subfigure}{.49\textwidth}
    \centering
    \includegraphics[width=0.8\linewidth]{circle_vectorfield.pdf}
  \end{subfigure}
  \caption{Illustration of tangent vectors in $\I$. The left illustration shows a path in the space of curves and how to obtain a tangent vector from this. The right illustration shows how to think of this tangent vector as a vector field on the circle.}
  \label{fig:def-tang-imm}
\end{figure}

We want to make our space of curves into a Riemmanian manifold so the next step is to impose a metric
\hl{[why do we/others call it a metric -- shouldn't it be an inner product?]}
on the tangent spaces. Because the tangent spaces are function spaces, the most obvious metric to use is some version of the $\mathrm{L}^2$ metric. 

\begin{definition}
  The \textit{$\mathrm{G}$ metric} or \textit{$\mathrm{G}_c$ metric} at the point $c \in \I$ is defined as
  \begin{equation*}
    \mathrm{G}_c(h,k) := 
    \left(
      \int_{\S^{1}} \left\langle{h(\theta)
          , k(\theta)}\right\rangle |c_{\theta}| \diff \theta
    \right)^{\frac{1}{2}},
  \end{equation*}
  $h,k \in T_cB = C^{\infty}(\S^1,\R^2)$.
\end{definition}

Adding the parameter derivative $c_{\theta}$ makes the metric invariant to reparametrizations which is essential when we want to let this fall down as a metric on $\mathcal{I}$. From this definition it is straightforward to define a notion of \textit{length} of a path in $\I$ which again allows us to define a \textit{distance} between two point in our space of curves. By defintion, the pointwise time derivative at $t$ of a path $t \mapsto q(t, \blank)$ in $\I$ is a tangent vector to the curve $q(t,\blank)$, so the following definition makes sense.

\begin{definition}
  \label{def:length-in-imm}
  Let $t \mapsto q(t,\blank)$ be a path in $\I$ with $t \in [0,1]$. The \textit{length} of the path $q$ with respect to the $\mathrm{G}$ metric  is
  \begin{equation*}
    L(q) := \int_{0}^{1} \mathrm{G}_{q(t,\blank)}(q_t,q_t) \diff t =
    \int_{0}^{1}
    \left(
      \int_{\S^{1}} \|q_t\|^2 |q_{\theta}| \diff \theta
    \right)^{\frac{1}{2}}
    \diff t.
  \end{equation*}
  The \textit{geodesic distance} \hl{[geodesic?]} between to curves $b,c \in \I$ (with respect to the $\mathrm{G}$ metric) is
  \begin{equation*}
    D(b,c) := \inf_{q \in \mathcal{Q}} L(q),
  \end{equation*}
  where $\mathcal{Q}$ denotes all paths $q$, such that $q(0,\blank)=b$ and $q(1,\blank)=c$.
\end{definition}


\subsection*{The manifold of unparametrized curves -- $\mathcal{I}(\S^1, \R^2)$}
\label{sec:manif-unpar-curv}

Using the setup from the previous section, we now define a manifold structure on $\mathcal{I}$. As it is easiest to represent element of this space with element from $\I$, we are particularly interested in how to calculate the length of a path in $\mathcal{I}$ directly from a parametrized representative of this path in $\I$. For a parametrized curve $c \in \I$ we write $\pi(c) := \mathrm{Im}(c)\in \mathcal{I}$ for the projection onto the quotient space, and we refer to $c\in\I$ as a \textit{(parametrized) representative} for $\pi(c) \in \mathcal{I}$. Similarly for a path $q = q(t, \blank) \in \I$ we construct the projection $\pi(q) = \pi(q(t,\blank)) \in \mathcal{I}$ and refer to $q$ as a representative for $\pi(q)$.

First of all we need to know how the tangent vectors of the quotient space $\mathcal{I}$ look like. We cannot directly use the same approach as in the previous subsection, because our definition of tangent vector would then be 
sensitive to reparametrizations: Consider a time-dependent
reparametrization $\phi(t,\theta)$, or, equivalently, a path $t \mapsto \phi(t,
\blank)$ in $\text{Diff}(\S^1)$; then the two paths $t \mapsto
\pi(q(t, \blank))$ and $t \mapsto \pi(q(t, \phi(t, \blank)))$ are
identical in $\mathcal{I}$ but give rise to two different vector fields.

To make better sense of the tangents vectors of the quotient space, we
use the following result.

\begin{proposition}
  \label{prop:horizontal-path}
  For every path $t \mapsto q(t, \blank)$ in $\mathrm{Imm}$ there exists a
  time-dependent reparametrization $t \mapsto \phi(t, \blank) \in
  \mathrm{Diff}(\S^{1})$ such that the path
  \begin{equation*}
    t \mapsto \tilde{q}(t, \theta):=q(t, \phi(t,\theta))
  \end{equation*}
  fulfills
  $\langle \tilde{q}_t, \tilde{q}_{\theta}\rangle=0$ for all $(t,\theta) \in [0,1]\times \S^1$, and such that $\phi(0, \theta)=\theta$. Furthermore, it holds that 
  \begin{equation}
    \label{eq:canon-repar}
    \phi_t = a \circ \phi
    = -\frac{\langle q_t \circ \phi, q_{\theta} \circ \phi\rangle}{|q_{\theta}\circ \phi|^2},
    \quad a := -\frac{\langle q_t,
      q_{\theta}\rangle}{|q_{\theta}|^2}.
  \end{equation}
\end{proposition}

\begin{proof}
  \hl{todo or ref.}
\end{proof}

\begin{remark}
  \label{remark:ortho-decom}
  \begin{enumerate}
  \item  When we write $\tilde{q}$ in the following, we shall we refer to a path obtained from another path $q$ by reparametrizing with $\phi$ above. 
  \item Note that for every vector field $h \in T_c(\I)$, determined from the path $q$, we can make a pointwise decomposition of $h$ onto $q_{\theta}(0, \blank)$ and $iq_{\theta}(0, \blank)$ by using the pointwise orthogonal projection. Explicitly we have that
  \begin{equation*}
    h = q_t = p_{q_{\theta}}(q_{t}) + p_{iq_{\theta}}(q_{t}), 
  \end{equation*}
  where $p$ is taken to be the standard \textit{pointwise} $\R^2$ orthogonal projection, which is given as
  \begin{equation*}
    p_v(u) = \frac{\langle v, u \rangle}{|v|^2} v, \quad u, v \in \R^2.
  \end{equation*}
  More correctly we should thus write
  \begin{equation*}
    h(\theta) = q_t(0,\theta) = p_{q_{\theta}(0,\theta)}(q_{t}(0,\theta)) +
    p_{iq_{\theta}(0,\theta)}(q_{t}(0,\theta)).
  \end{equation*}
  From this we see that the time derivative of the reparametrization in the previous Proposition is the coefficient function for the projection onto the parameter derivative of the original path $q$; this becomes relevant in a moment.
\end{enumerate}
\end{remark}

We can use this result to define tangent vectors to elements of $\mathcal{I}$ in
a consistent way:

\begin{definition}
  \label{def:tang-quotient}
  A \textit{tangent vector} $h$ to an element $\pi(c) \in \mathcal{I}$ is defined as a vector fields obtained from some path $(-\epsilon, \epsilon) \ni t \mapsto q(t, \blank) \in \I$, with $q(0,\blank) = c$, by
  \begin{equation}
    \label{eq:tang-quotient}
    h(\theta) = \frac{\partial }{\partial t} \bigg\vert_{t=0} \tilde{q}(t,\theta)
    = \frac{\partial }{\partial t} \bigg\vert_{t=0} q(t, \phi(t,\theta)),
  \end{equation}
  with $\tilde{q}$ and $\phi$ given in accordance with remark~\ref{remark:ortho-decom}.

  \hl{[NB: Does this actually solve the problem about define the ubiquity in defining tangent vectors? Not clear that applying $\phi$ to a reparametrization of $q$ will yield the same result? However, it shows that we can always think of are path as moving orthonormally; but maybe we should compine this definition with the proposition below?]}
\end{definition}

First we note that this gives us the following visualization of the tangents spaces of $B$.

\begin{proposition}
The tangent space to an element $\pi(c) \in \mathcal{I}$ consists of orthonormal vector fields on the circle, i.e.,
  \begin{equation*}
    T_{\pi(c)}(\mathcal{I}) = 
    \left\{
      b i c_{\theta} \mid b \in C^{\infty}(\S^1,\R)
    \right\}.
  \end{equation*}
\end{proposition}

\begin{proof}
  This follows from Definition~\ref{def:tang-quotient} and the property of the reparametrization $\phi$. 
\end{proof}

As the length of a path in $\mathcal{I}$ is our primary concern, we skip straight to this without actually defining the inner product on the tangent spaces. The central idea is to mimic Definition~\ref{def:length-in-imm} on the reparametrized path $\tilde{q}$; and though we don't bother to go trough a inner product on the tangent spaces, we note that by this construction the time derivative of the path $\tilde{q}$ is a valid tangent vector in $\mathcal{I}$ at every point $\pi(q(t,\blank))$

\begin{definition}
  The \textit{length} in $\mathcal{I}$ of a path $\mathrm{q}=\pi(q)$ is
  \begin{equation*}
    \mathcal{L}(\mathrm{q})= \mathcal{L}(\pi(q)) := L(\tilde{q}) = 
    \int_{0}^{1}
    \left(
      \int_{\S^{1}} \|\tilde{q}_t\|^2 |\tilde{q}_{\theta}| \diff \theta
    \right)^{\frac{1}{2}}
    \diff t,
  \end{equation*}
  with $\tilde{q}(t,\theta)=q(t,\phi(\theta))$ as in remark~\ref{remark:ortho-decom}. 
  The \textit{geodesic distance} between to shapes $\mathrm{b}, \mathrm{c} \in \mathcal{I}$ represented by $c,b \in \I$, is
  \begin{equation*}
    \mathcal{D}(\mathrm{b},\mathrm{c}) = \mathcal{D}(\pi(b),\pi(c)) := \inf_{q \in \mathrm{Q}} \mathcal{L}(q),
  \end{equation*}
  where $\mathrm{Q}$ denotes all paths $\mathrm{q}$ in $\mathcal{I}$, such that $\mathrm{q}(0)=\pi(b)$ and $\mathrm{q}(1)=\pi(c)$. 
\end{definition}

\begin{proposition}
    $\mathcal{L}$ is well-defined, and for any representative $t \mapsto q(t, \blank) \in \I $ of the path $t \mapsto \tilde{q}(t) \in B$ the length can be calculated as
    \begin{equation}
      \label{eq:length-quotient}
    \mathcal{L}(\tilde{q}) = \int_{0}^{1} 
    \left(
      \int_{\S^1}  \frac{\langle q_t, i q_{\theta}\rangle^2}{|q_{\theta}|} \diff \theta
    \right)^{\frac{1}{2}} \diff t.
  \end{equation}
\end{proposition}

\begin{proof}
  For ease of notation, write $q \circ \phi$ to mean $q(t, \phi(t,\theta))$ and so on during this proof.
  First, we shows that \eqref{eq:length-quotient} implies that $\mathcal{L}$ is well-defined; so assume \eqref{eq:length-quotient} holds and let $q(t,\blank)$ and $p(t, \blank)$ be two different representatives for $\tilde{q}(t)$. This means that we must have a reparametrization $\psi(t,\theta)$ such that
  \begin{equation*}
    p(t, \psi(t,\theta)) = q(t, \theta).
  \end{equation*}
  Then
  \begin{equation*}
    p_t = q_t\circ \psi + \psi_t (q_t \circ \psi), \quad
    p_{\theta} = \psi_{\theta}(q_{\theta} \circ \psi),
  \end{equation*}
  so
  \begin{equation*}
    \begin{aligned}
      \left\langle
        p_t, i p_{\theta}
      \right\rangle
     &  =     \left\langle
        q_t \circ \psi + \psi_t (q_{\theta} \circ \psi),
        \psi_{\theta} (iq_{\theta} \circ \psi)
      \right\rangle \\
     & =     \left\langle
        q_t \circ \psi,
        \psi_{\theta} (iq_{\theta} \circ \psi)
      \right\rangle \\
      & = (\langle q_t, i q_{\theta}\rangle \circ\psi) \psi_{\theta},
    \end{aligned}
  \end{equation*}
  and thus
  \begin{equation*}
    \int_{\S^1} \frac{\langle p_t, i p_{\theta}\rangle^2}{|p_{\theta}|} \diff \theta
    =
    \int_{\S^1} 
    \left(
      \frac{\langle q_t, i q_{\theta}\rangle^2}{|q_{\theta}|}
    \right) \circ \psi |\psi_{\theta}| \diff \theta
    =    \int_{\S^1} 
      \frac{\langle q_t, i q_{\theta}\rangle^2}{|q_{\theta}|}  \diff \theta,
    \end{equation*}
    which shows that the length does not depend on the parametrization of the path.

  Next, by construction, the tangent vectors along the path $\tilde{q}$ in $B$ is given as
  \begin{equation*}
    \frac{\partial }{\partial t}  (q \circ \phi)
    = q_t \circ \phi + \phi_t (q_{\theta} \circ \phi) 
  \end{equation*}
  Now, as in remark~\ref{remark:ortho-decom}, decompose $q_t \circ \phi $ by projecting pointwise onto $q_t \circ \phi $ and $i q_t \circ \phi $. Then we get
  \begin{equation*}
    q_t \circ \phi =
    % \frac{\langle q_t \circ \phi, iq_{\theta} \circ \phi \rangle}
    % {|q_{\theta} \circ \phi|^2}(iq_{\theta} \circ \phi) +
    % \frac{\langle q_t \circ \phi, q_{\theta} \circ \phi \rangle}
    % {|q_{\theta} \circ \phi|^2}(q_{\theta} \circ \phi),
    % =
    \left(
      \frac{\langle q_t  , iq_{\theta}   \rangle}
    {|q_{\theta}  |^2}iq_{\theta}  
  \right) \circ \phi +
  \left(
    \frac{\langle q_t  , q_{\theta}   \rangle}
    {|q_{\theta}  |^2}q_{\theta} 
  \right) \circ \phi,
  \end{equation*}
  and by Proposition~\ref{prop:horizontal-path} we see that the last term cancels with
  $\phi_t(q_{\theta}\circ \phi)$, so
  \begin{equation*}
    \frac{\partial }{\partial t} ( q \circ \phi )=
    % \frac{\langle q_t \circ \phi, iq_{\theta} \circ \phi \rangle}
    % {|q_{\theta} \circ \phi|^2}(iq_{\theta} \circ \phi)
    % =   
    \left(
      \frac{\langle q_t  , iq_{\theta}   \rangle}
      {|q_{\theta}  |^2}iq_{\theta} 
    \right)
    \circ \phi.
  \end{equation*}
  For any fixed $t \in [0,1]$, the reparametrization $\phi$ is just an ordinary reparametrization of the curve $\theta \mapsto q(t,\theta)$, so by invariance of the metric we have that
  \begin{equation*}
    \begin{aligned}
    & G_{q(t,\phi(t,\blank))}^2
    \left( \frac{\partial }{\partial t} ( q \circ \phi ),
      \frac{\partial }{\partial t} ( q \circ \phi )
  \right) \\
  & \quad=  G_{q(t,\phi(t,\blank))}^2
    \left(
          \left(
      \frac{\langle q_t  , iq_{\theta}   \rangle}
      {|q_{\theta}  |^2}iq_{\theta} 
    \right)
    \circ \phi,
        \left(
      \frac{\langle q_t  , iq_{\theta}   \rangle}
      {|q_{\theta}  |^2}iq_{\theta} 
    \right)
    \circ \phi
    \right) \\
    & \quad =
    G_{q(t,\blank)}^2
    \left(
      \frac{\langle q_t  , iq_{\theta}   \rangle}
      {|q_{\theta}  |^2}iq_{\theta} ,
      \frac{\langle q_t  , iq_{\theta}   \rangle}
      {|q_{\theta}  |^2}iq_{\theta} 
    \right) \\
    & \quad =
    \int_{\S^1} 
    \left\|
      \frac{\langle q_t  , iq_{\theta}   \rangle}
      {|q_{\theta}  |^2}iq_{\theta}
    \right\|^2 |q_{\theta}  | \diff \theta \\
    & \quad =
    \int_{\S^1} 
      \frac{\langle q_t  , iq_{\theta}   \rangle^2}
      {|q_{\theta}  |}  \diff \theta,
  \end{aligned}
\end{equation*}
from which \eqref{eq:length-quotient} follows immediately by definition. 
\end{proof}

\hl{Some concluding text...}

\subsection*{\hl{Comparing this to the formal construction as a Frechet manifold}}
\label{sec:hlcomp-this-form}

\hl{... a lot of references ...}

\subsection*{The L2 metric vanishes}
\label{sec:l2-metric-vanishes}

\hl{Some introductory text...}

Definition of the length of the path $\phi$ directly from the path alone:
\begin{equation*}
  L(\phi) = \int_{0}^{1} \int_{\S^{1}}
  \frac{\langle\phi_{t},i \phi_{\theta}\rangle^2}{|\phi_{\theta}|}  \diff \theta \diff t
\end{equation*}
Rewrite this as
\begin{equation*}
  \int_{0}^{1} \int_{\S^{1}}
  \left(
    \frac{\langle\phi_{t},i
      \phi_{\theta}\rangle}{|\phi_{t}||\phi_{\theta}|}
  \right)^2
  |\phi_t|^2   |\phi_{\theta}|
  \diff \theta \diff t
  =
  \int_{0}^{1} \int_{\S^{1}}
  \cos(\alpha(\phi_t, i\phi_{\theta}))^2
  |\phi_t|^2   |\phi_{\theta}|
  \diff \theta \diff t,
\end{equation*}
with $\alpha(x,y)$ denoting the angle between $x$ and $y$. When
constructing a zigzag-path the angle will be \hl{(approximately?)}
constant in $\theta$ and $t$,
and it is given by
\begin{equation*}
  n = \tan(\alpha).
\end{equation*}
We have that
\begin{equation*}
  \cos(\arctan(n)) = (1+n^2)^{-1/2}
  = O(n^{-1}),
\end{equation*}
so we can write
\begin{equation*}
  L(\phi) = O(n^{-2})
  \int_{0}^{1} \int_{\S^{1}}
  |\phi_t|^2   |\phi_{\theta}|
  \diff \theta \diff t,
\end{equation*}
in this case. To show that such a zigzag path has arbitrary
small length we just need to show that the integral does not grow
faster than $n^2$.

As an example, take the simply case where we expand the circle $e^{i\pi\theta}$ to
$2e^{i\pi\theta}$. The zigzag path in then concretely given as
\begin{equation*}
  \phi(t,\theta) = e^{i\pi\theta}
    \sum_{k=0}^{n-1}
    h^{n,k}(t,\theta) + g^{n,k}(t,\theta)
\end{equation*}
where
\begin{equation*}
  \begin{aligned}
    h^{n,k}(t,\theta) & := 1_{[\frac{k}{n},\frac{k}{n} +
      \frac{1}{2n})}(\theta) \left( 1+2t(n\theta-k) \right), \\
    g^{n,k}(t,\theta) & := 1_{[\frac{k}{n} + \frac{1}{2n},\frac{k+1}{n})}(\theta)
    \left( 1+2t(1-n\theta-k) \right).
  \end{aligned}
\end{equation*}

We have that
\begin{equation*}
  \begin{aligned}
    |\phi_t| & = \sum_{k=0}^{n-1} |h^{n,k}_t| + \sum_{k=0}^{n-1}
    |g^{n,k}_t|, \\
    |\phi_{\theta}| & = \sum_{k=0}^{n-1} |h^{n,k}_{\theta} + h^{n,k} | +
    \sum_{k=0}^{n-1}
    |g^{n,k}_{\theta} + g^{n,k} |,
  \end{aligned}
\end{equation*}
so by symmetry
\begin{equation*}
  \begin{aligned}
    \int_{0}^{1}
    |\phi_t|^2   |\phi_{\theta}|
    \diff \theta
    & =
    2n \int_{0}^{\frac{1}{2n}} |h^{n,0}_t|^2 |h^{n,0}_{\theta} + h^{n,0} |
    \diff \theta \\
    & = 2n \int_{0}^{\frac{1}{2n}}
    (2n\theta)^2(2tn+1+t2n\theta) \diff \theta \\
    & = \int_{0}^1
    u^2(2tn+1+t u) \diff \theta \\
    & = O(n),
  \end{aligned}
\end{equation*}
for $t\in[0,1]$ which gives the result.




\begin{comment}

\section{Question / considerations / imprecisions}
\label{sec:quest--cons}

Above we made an intuitively reasonable construction, but ignored some
of technicalities, which we list here.
\begin{itemize}
\item Equivalence classes: We should define the tangent vectors as
  equivalence classes of path in $B$; this should all determine a
  unique vector field.
\item The manifold structure of $B$: We simply used the intuitive idea
  to differentiate in ``time'' for each fixed point on a curve,
  $q_t(x)$. However, as we have not specified the chart for the
  manifold $B$, it is not obvious that this construction corresponds
  to the one made in the finite dimensional case. Technically we
  would need a chart $\phi \colon U \rightarrow F$, with $F$ some
  Fr\'echet space and then show some sort of Fr\'echet-differentiability of the
  composite function $\phi \circ \gamma$.
\item We mentioned that in the finite dimensional case the two
  definitions (through derivates and tangents to curves, respectively)
  are equivalent. It is not obvious that this also hold in the
  infinite dimensional case.
\item At the beginning we eliminated the knowledge of the
  parametrization of a curve $q \in B$ to make the definition of $B$
  simpler. However, we actually use a parametrization later, and thus
  we should make sure that reparametrizations does not matter for the
  construction of the tangent space. (It does not, as it would just
  move the vector field around $\S^1$ according to the
  reparametrization.)
\item Are there any smoothness assumptions (or something) about vector
  field we need to validate? For example, just the fact that the map
  $t \mapsto q_t(x)$ behaves nice does of course \textit{not} imply
  that also the derived vector field $\gamma'_q$ behaves nicely in
  $x$ -- which is what we would need to get a smooth vector field
  (?).
\item Is it correctly formulated that we need to think of $\S^1
  \subset \R^2$ to make sense of a vector field on the circle?
\end{itemize}

\section{Sort of the same construction by the argument from Mumford and Michor}
\label{sec:defin-metr-tang}

\hl{This is the sort of argument they (Mumford and Michor) give in the
  beginning of section 3.2. Not sure how to exactly finish this.}

Consider the following metric on $\I$.

\begin{definition}[The L2 metric on $\I$]
  The \textit{L2 metric} $G^2_c$ at the point $c \in \I$ is defined as
  \begin{equation*}
    G^2_c(h,k) := \int_{\S^{1}} \left\langle{h(\theta)
    , k(\theta)}\right\rangle |c_{\theta}| \diff \theta,
  \end{equation*}
  $h,k \in T_cB = C^{\infty}(\S^1,\R^2)$.
\end{definition}

Adding the parametrization derivative $c_{\theta}$ makes this
invariant under \hl{(orientation preserving?)}
reparametrizations. Because of that, this metric also induces a metric
on the quotient space $B$, which we can identify as follows. Consider
first the tangent space at $c$ to the \textit{orbit} of the diffeomorphism
group
\begin{equation*}
  T_c(c \circ \text{Diff}(\S^{1})), \quad c \circ
  \text{Diff}(\S^{1}) := 
  \left\{
    c \circ \phi \mid \phi \in \text{Diff}(\S^{1})
  \right\} .
\end{equation*}
A path in the orbit $c \circ \text{Diff}(\S^{1})$ reduces to
\begin{equation*}
  t \mapsto 
  \left(
    \theta \mapsto (c \circ \phi_t) (\theta)
\right).
\end{equation*}
Differentiating this with respect to $t$ shows that we can identify
\begin{equation}
  \label{eq:tang-orbit}
  T_c(c \circ \text{Diff}(\S^{1})) = 
  \left\{
    g c_{\theta} \mid g \in C^{\infty}(\S^1, \R)
  \right\}.
\end{equation}
The \textit{normal space} $\mathcal{N}_c \subset T_c(\I)$ consists of
tangent vectors that are orthogonal to $T_c(c \circ
\text{Diff}(\S^{1}))$ with respect to $G_c^2$. As the projection
\begin{equation*}
  \pi \colon \I \rightarrow B = \I/\text{Diff}(\S^{1})
\end{equation*}
sends the whole orbit to a single point \hl{it follows} that tangents
vectors to the projected element $\pi(c) \in B$ can essentially
be identified with elements of the normal space $\mathcal{N}_c$.
From \eqref{eq:tang-orbit} we see that
\begin{equation*}
  \mathcal{N}_c =
  \left\{
    g i c_{\theta} \mid g \in C^{\infty}(\S^1, \R)
  \right\},
\end{equation*}
with $i c_{\theta} = |c_{\theta}| n_c$, $n_c$ being the unit normal
vector field along $c$. Finally, for a given $h \in T_c(\I)$
we can explicitly find the decomposition of $h$ into the two orthogonal
subspaces of the tangent space by projecting $h$ onto $c_{\theta}$ and
$ic_{\theta}$, which gives
\begin{equation*}
  \begin{aligned}
    h &  = p_{c_{\theta}}(h) + p_{ic_{\theta}}(h)
    \in T_c(c \circ \text{Diff}(\S^{1})) \oplus \mathcal{N}_c
    ,\\
    p_{c_{\theta}}(h) & = 
    \frac{\langle h, c_{\theta}\rangle}{|c_{\theta}|^2} c_{\theta}
    \in T_c(c \circ \text{Diff}(\S^{1})), \\
    p_{ic_{\theta}}(h) & = 
    \frac{\langle h, ic_{\theta}\rangle}{|c_{\theta}|^2} ic_{\theta}
    \in \mathcal{N}_c,
  \end{aligned}
\end{equation*}
with $p_{v}(u)$ denoting the standard orthogonal projection in $\R^2$.

This allows us to calculate a formula for
the induced metric on $B$. In particular, it allows us to explicitly calculate
the length of a projected path $t \mapsto \pi(q(t,\blank)) \in B $
where $t \mapsto q(t,\blank) \in \I$ is a path in $\I$.
In this setting, at each time point $t$
the tangent vector is $q_t$ so \hl{by the above arguments} we get that
the induced metric is
\begin{equation*}
  \begin{aligned}
    L^2(\pi(q))& = \int_{0}^{1}
    G^2_{q(t,\blank)}(p_{iq_{\theta}}(q_t),p_{iq_{\theta}}(q_t)) \diff t \\
    & = \int_{0}^{1}
    \int_{\S^1} 
    \left\langle
      \frac{\langle h, iq_{\theta}\rangle}{|q_{\theta}|^2} iq_{\theta},
      \frac{\langle h, iq_{\theta}\rangle}{|q_{\theta}|^2} iq_{\theta}
  \right\rangle | q_{\theta}|\diff \theta \diff t.
  \end{aligned}
\end{equation*}

\section{Short construction of tangent vectors from alternative definition}
\label{sec:short-constr-tang}

Lee defines the tangent
space at a point $p \in B$ through the notation of
\textit{derivations};

Using this, one can define the notion of a
tanget vector to a path in $B$ passing through the point $p$. Then,
one can define an equivalence relation on the space of such paths and
obtain an equivalent definition of the tanget space, which is more
intuitive. One can also work the other way around and start by
defining the notion of tanget vector to paths in $B$ (as is done on
Wiki). We briefly do that here:

For a neighbourhood $U \subset B$ containing $p$ we have a smooth coordinate
chart $\phi\colon U \rightarrow \R^n$. For a path $\gamma \colon
[-\epsilon,\epsilon] \rightarrow B$ with $\phi(0) = p$, it makes perfect sense to consider
differentiability of the map $\phi \circ \gamma \colon
[0,1]\rightarrow \R^n$. Now, the relation
\begin{equation*}
  \gamma_1 \sim \gamma_2 \iff
  (\phi \circ \gamma_1)'(0) = (\phi \circ \gamma_2)'(0),
\end{equation*}
defines an equivalence relation on the collection of all such paths
$\gamma$. An equivalence class of such paths, denoted by $[\gamma'_p]$ (or simply
$\gamma'_p$), is called a \textit{tangent vector}
at the point $p$. The collection of all tanget vectors make up the
tanget space $T_pB$ at $p$.

\end{comment}


\end{document}





